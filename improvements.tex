Každou aplikaci je možné zlepšovat. Naše není výjimkou. V~této části konkrétněji popíšeme možná vylepšení aplikace,
které mohou usnadnit testování a pomohou při vývoji nebo zlepší celkové chování aplikace. Za nejdůležitější si dovolíme označit
automatické testování backendu pomocí integračních testů. Zde si ukážeme menší ukázku, jak by takové testování mělo vypadat.

\section{Testování backend části}

Na backendu je potřeba otestovat hlavně jeho \gls{rest} \gls{api} metody, které se používají ke komunikaci. Pro jednoduché
testování v rámci vývoje aplikace bylo využito nástroje Postman \cite{postmanSW}, tento nástroj ale není použitelný pro rozsáhlé
a repetitivní testování. Proto se podíváme na využití jiného programu. Tím bude \textit{pytest}. Jde o rozšíření pro jazyk Python,
který dovoluje jednoduché definování automatických testů.

Než napíšeme první test v \textit{pytest}, je potřeba nastavit \textit{test fixtures}. \textit{Test fixture} je základní část
testu, která je pro všechny testy stejná. \textit{Test fixture} například představuje připojení k databázi, nebo může posloužit k načtení
celé aplikace v rámci integračních testů. Zde je menší ukázka jak v naší aplikaci vypadá taková \textit{test fixtures} \ref{lst:testFixture}.

Pro testování backendu by mělo být použito integrační testování, kdy budeme testovat správné odpovědi na jednotlivé \gls{http} dotazy z našeho
testovacího prostředí. Před samotným dotazem provedeme patřičné nasazení dat, na kterých provedeme testování naší odpovědi. Při této metodě
testování se provádí test celého funkčního celku. Existuje připojení na databázi a testovací
klient, který se připojuje na testovací server. Zde je ukázka takového testu na příkladu s testem přihlášení \ref{lst:authTest}. V této ukázce
je vidět volání funkce \texttt{test\_login}. Parametry této funkce jsou \textit{test fixtures}. Prvním je náš testovací klient, který umí odeslat
\gls{http} dotaz na backend, druhým je připojení k testovací databázi, můžeme tedy přímo vkládat data do databáze, díky tomu také máme
možnost kontrolovat správnost dat přímo v~databázi a zajistit tak správnost výsledku.

\begin{listing}[h]
    \begin{minted}[linenos,breaklines]{python}
# Ukázka z test/conftest.py
import pytest

from moddoc import create_app


@pytest.fixture(scope='session')
def app():
    """
    Creates instance of application with test configuration for each test session
    """
    app = create_app('test')

    # register our blueprints
    with app.app_context():
        app.init_api()

    # return generator for app
    yield app
    \end{minted}
    \caption{Ukázka test fixture}
    \label{lst:testFixture}
\end{listing}

\begin{listing}[h]
    \begin{minted}[linenos,breaklines]{python}
# ukázka  test/test_auth.py
from moddoc.model import User


def test_login(client, db):
    # prepare data
    username = 'test'
    password = 'SuperSecret1'
    user = User(username=username, email='test@example.com', password=password)
    db.session.add(user)
    db.session.commit()

    # create data
    data = {'username': username, 'password': password}

    # get server response
    response = client.post('auth/login', json=data)

    assert response.status_code == 200
    assert response.get_json()['access_token'] is not None
    \end{minted}
    \caption{Integrační test příhlášení}
    \label{lst:authTest}
\end{listing}

Díky těmto testům můžeme snadno a rychle pokrýt celou funkcionalitu backendu, a můžeme zaručit správné zpracování dat ze strany backendu.
Ovšem tato metoda testování nám v případě chyby v některé z části kódu neukáže hned přesné místo, kde problém vznikl. S tím by nám pomohly
\textit{unit} testy, neboli testy jednotlivých funkcí. Tyto testy se vyznačují tím, že testují pouze danou funkci či metodu. Pokud se v testované funkci
vyskytuje volání další funkce, je třeba toto volání takzvaně \textit{mockovat}. \textit{Mockování} je nahrazování původního volání novým voláním a
lze takto nahradit například komunikaci s databází, která by v rámci unit testů neměla být využita. \textit{Unit testování} je dalším dobrým
rozšířením pro naší aplikaci.

\clearpage

\section{Integrace lepšího verzování}

V hodnocení aplikace jsme již na toto téma narazili. Nyní si řekněme, jak by takové řešení mělo vypadat. Možností jak toto udělat, je několik,
ale asi nejjednodušší verzí se zdá být integrace příkazů z \textit{gitu} (\textit{git} je implementace \gls{vcs}) přímo v našem kódu a možnost propojení buď s veřejným repozitářem
anebo vlastním hostovaným \gls{vcs} serverem. S tímto je také spojená nutnost změnit princip ukládání modulů i dokumentů a také vytvoření nové
logiky pro vytváření předchozích verzí, kdy by se verze orientovaly podle příslušných verzí z verzovacího systému \textit{git}.

\section{Možnost online náhledu}

Dalším možným rozšířením pro naší aplikaci by byla možnost náhledu dokumentu a modulů ve fázi úprav. Uživatel by tak měl možnost
jednodušší kontroly formátování výsledného dokumentu nebo modulu. Pro vytváření těchto náhledů by bylo možné využít \textit{docutils},
které jsme již zmínili v analýze nástrojů. Zároveň s náhledem by bylo možné také definovat vlastní úpravy pro styl, ve kterém
bude výsledný dokument generován.