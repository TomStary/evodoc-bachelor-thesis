\section{Lineární přístup}

Většina dnes psaných dokumentů vzniká stále stejným způsobem jako posledních pár století, ani moderní technologie tento přístup moc nezměnili.
Dokumenty vznikají lineárně, nejdříve se vytvoří návrh také nazývaný jako draft, který se poté dá ke kontrole dalším editorům. Ti se zaměří
na jednotlivé části tvořící dokument, kontroluje se pravopis, stylistika a obsah.

Po všech kontrolách se dokument ještě může vrátit autorovi ke konečné kontrole či korekci. Dále poté putuje nakladateli, který dokument,
knihu či odbornou publikaci distribuje. Celý tento postup je znázorněn na grafu pod textem.
