\chapter{Nejdříve trocha historie TODO: možná lepší název kapitoly}

Pokud se chceme zabývat psaním dokumentů, měli bychom si nejdříve udělat menší výlet do historie. V dnešní době bereme vytváření
jakýkoliv dokumentů jako samozřejmost, stačí nám k tomu tužka a kus papíru, nebo můžeme použít počítač, či dokonce i chytrý telefon.

Základem každého dokumentu je písmo, písmo jako takové se prvně objevuje už v 7. tisíciletí před naším letopočtem a to v číně,
kde se našli kosterní pozůstatky a blízko nich i krunýře želv, na kterých se našlo první písmo. \cite{EarliestWriting} Poté se písmo rozšiřuje
i ve staré Mezopotámie a ve starém Egyptě, kde se poté také objevuje jeden z prvních přechůdců papíru papyrus. Ten ovšem po čase upadl
v zapomění a nahradil jej pergamen, který se používal až do 18. století, přestože se v období 13. až 14. století objevuje jeho konkurence, papír.

Po celou tuto dobu se k psaní textů používala výhradně, na území Evropy, lidská síla, tedy všechny dokumenty byly psány ručně, pokud chtěl člověk například
kopii oné knihy, bylo nutné najít písaře, který by danou knihu opsal. Toto byl také jeden z důvodů proč nedocházelo k masovému šíření jakýkoliv textů a byla
to výhrada pro bohatší vrstvy společnosti.