\section{Nejdříve trocha historie}\todo{předělat nadpis historie}

Pokud se chceme zabývat psaním dokumentů, měli bychom si nejdříve udělat menší výlet do historie. V dnešní době bereme vytváření
jakýkoliv dokumentů jako samozřejmost, stačí nám k tomu tužka a kus papíru, nebo můžeme použít počítač, či dokonce i chytrý telefon.

Základem každého dokumentu je písmo, písmo jako takové se prvně objevuje už v 7. tisíciletí před naším letopočtem a to v číně,
kde se našli kosterní pozůstatky a blízko nich i krunýře želv, na kterých se našlo první písmo. \cite{EarliestWriting} Poté se písmo rozšiřuje
i ve staré Mezopotámii a ve starém Egyptě, kde se poté také objevuje jeden z prvních předchůdců papíru papyrus. Ten ovšem po čase upadl
v zapomění a nahradil jej pergamen, který se používal až do 18. století, přestože se v období 13. až 14. století objevuje jeho konkurence, papír.

Po celou tuto dobu se k psaní textů používala výhradně, na území Evropy, lidská síla, tedy všechny dokumenty byly psány ručně, pokud chtěl člověk například
kopii oné knihy, bylo nutné najít písaře, který by danou knihu opsal. Toto byl také jeden z důvodů proč nedocházelo k masovému šíření jakýkoliv textů a byla
to výhrada pro bohatší vrstvy společnosti.

Toto se ovšem změnilo s příchodem knihtisku, o který se hlavně zasloužil Johan Gutenberg, který v polovině 15. století významně zdokonalil knihtisk,
který se pak do konce století rozšířil po celé Evropě. To přineslo zlevnění a zrychlení tvorby nových děl a také se knihy začli šířit mezi větší skupinu
obyvatelstva. Ovšem toto rozšíření stále nebylo tak velké jaké máme dnes a to hlavně díky negramotnosti nižších tříd. V 19. století se s rozšiřující gramotností
i nižších tříd objevuje stále větší hlad po tiskovinách, s průmyslovou revolucí se také rozšiřují možnosti tisku, například roku 1799 si nechal N. L. Robert
patentovat vynález stoje na výrobu papíru v tzv. \uv{nekonečném} pásu. \cite{Papir}

V dnešní době ovšem máme každý možnost si doma vytvořit vlastní dokumenty, noviny či knihy, elektronickou cestou. Po příchodu internetu tak nyní máme možnost
své dokumenty psát a sdílet s ostatními a dokonce na nich i společně pracovat ve stejném okamžitu. Jejich reprodukce do fyzické formy také není problém a to díky
domácím tiskárnám, které jsou cenově dostupné. Při větším počtu kopií je možnost se obrátit i na specializované firmy, které jsou schopny velkoobjemových tisků
v různých kvalitách.

\todo{přidat historii psacího stroje}
\todo{Question: Přidat historii PC?}