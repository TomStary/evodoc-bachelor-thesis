Pokud se chceme zabývat psaním dokumentů, musíme se nejdříve podívat do historie, na to kdy a jak se začli psát první dokumenty, jak se psaní dokumentů
měnilo s tím jak se měnila vyspělost lidské civilizace.

V dnešní době má každý z nás možnost vytvořit dokument a sdílet jej s ostatními na celém světe a to všechno v rámci několika okamžiků. Toto ovšem nebylo
vždy možné. Podívejme se, jak jsme se jako lidstvo dostalo od vyškrabávání znaků do krunýřů až po dnešní psaní a sdílení dokumentu online.

\section{Od krunýřů ke knihtisku}

Základem každého dokumentu je písmo, písmo jako takové se prvně objevuje už v 7. tisíciletí před naším letopočtem a to v číně,
kde se našli kosterní pozůstatky a blízko nich i krunýře želv, na kterých se našlo první písmo. \cite{EarliestWriting} Toto je nejstarší doposud
nalezený artefakt obsahující písmo.

Písmo se o něco později také objevuje v Mezopotámii. Zde se objevuje klínové písmo, které přišlo jako zjednodušení pro piktogramy, které se používali
pro kontrolu obilí a dobytka. Společně s klínovým písmem se nezávysle na sobě rozvíjí písmo i na území starého Egypta, zde v podobě hieroglyfů. Tyto dvě
písma se pak rozšiřují po celém regionu, díky tomu roste v celém regionu efektivita ekonomiky a objevují se první historické záznamy
\cite{MesopotamiaHistory} a také první sepsané zákony.

V Mezopotámii se pro psaní klínového písma používaly hliněné destičky, kdežto ve starém Egyptě se používal papyrus. Papyrus se vyráběl ze stébel šáchoru
papírodárného a byl rozšířen po celém středomoří. Jeho výroba byla zapomenuta okolo roku 1100.

V evropě se poté rozšiřuje používání papíru, který se k nám dostal z Číny díky Arabům. Papír byl oproti pergamenům, které jsou vyráběny z kůže, levnější
na výrobu, ale byl horší kvality, tudíž pro důležité dokumenty se používal pergamen. Do poloviny 15. století se ovšem stále většina dokumentů psala bez
použití sofistikované techniky, vše bylo stále opisováno ručně. To ovšem změnil Johan Gutenberg, který významně zdokonalil knihtisk, který se pak do
konce století rozšířil po celé Evropě. Knihtisk zde byl již dříve, ale Johan přišel s nápadem odlití jednotlivých písmen z kovu, tím se zvýšila
jejich životnost a díky tomu se snížila cena celého tisku. Snížení ceny mělo za následek rozšíření knih mezí více lidí a tím i zvednutí gramotnosti
obyvatelstva. \cite{ucebnice1}

\section{Průmyslová revoluce}

Ještě před příchodem průmyslové revoluce se objevují první pokusy o psací stroj a to již v roce 1714, kdy byl v Anglii patentován stroj
\uv{vyrážející písmenka na papír} \cite{FirstTypewriter}, ovšem s psacími stroji měl pramálo společného. O více než 100 let později se
objevuje první psací stroj, který přinesl rozložení kláves, které známe i z dnešních moderních počítačů. Toto rozložení bylo zavedeno kvůli
problémům se zasekáváním znaků při stisku více znaků po sobě, které se nacházely blízko sebe. Psací stroje zde s námi byly po většinu 20. století
a byly nahrazeny až počítači, ale o těch až později. Krom psacích strojů se s průmyslovou revolucí se také rozšiřují možnosti tisku, například roku 1799
si nechal N. L. Robert patentovat vynález stoje na výrobu papíru v tzv. \uv{nekonečném} pásu. \cite{Papir}

\section{Moderní technologie}

\subsection{Software}