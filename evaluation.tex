\section{Zhodnocení přínosů systém}

Nyní je na čase zhodnotit přínosy našeho systému. Jednou z hlavních předností našeho systému je jeho
forma šíření, jedná se o OSS (open source systém), neboli otevřený software. Všechny zdrojové kódy
jsou veřejně dostupné a to na verzovacím portálu GitHub, kde si každý může stáhnout aktuální
verzi a to jak frontendu, tak i backendu. Každý si může vytvořit fork našeho kódu. Fork je vlastně
pouze kopírování verzovaného kódu do nového repozitáře. Díky tomu může kdokoliv vytvořit vlastní změny
a případně požádat o jejich zapracování do původního repozitáře. Mimo jiné i zdrojový kód této práce
se nachází na tomto portálu a je také veřejně přístupný.

Práce kromě svého primárního účelu, kterým je generování dokumentů, může také posloužit jako výukový materiál
pro začínající vývojáře. V práci i v implementaci jsou popsané základní postupy pro další rozvoj a při programování
této práce byl kladen důraz na co největší kvalitu kódu a také na jeho snadné pochopení a jednoduchou orientaci v něm.

\section{Porovnání s analýzou}

V analýze jsme sepsali seznam funkčních a nefunkčních požadavků pro náš systém. Podíváme se
jak jsme je v naší implementaci naplnili a případně, co nového jsme oproti analýze přidali. Ke každému požadavku
přiřadíme tu část aplikace, která naplňuje daný požadavek. U některých také dodáme, co jsme oproti analýze
přidali či odebrali.

\begin{enumerate}
    \item Správa modulů, možnost jejich úpravy či smazání.

          Tuto část jsme v rámci naší implementace splnili a dokonce jsme ji ještě doplnili o repozitáře,
          které nám zajistí jednodušší správu použitých modulů v dokumentech a také zjednodušší sdílení
          modulů.
    \item Generování dokumentů.

          Generování dokumentů jsme splnili v plném rozsahu podle námi definovaných specifikací.
    \item Verzování dokumentů i modulů.

          Verzování modulů i dokumentů jsme zajistili pomocí ukládání předcho\-zích stavů do databáze. Toto řešení
          ovšem bude trpět na zatížení daty. Čím více dat budeme mít, tím horší můžeme očekávat výkon. Pro zlepšení
          této situace by bylo třeba implementovat \gls{vcs} pro verzování dokumentů i modulů.
    \item Modularita aplikace.

         Z pohledu frontendu nám s modularitou aplikace velmi pomůže samotný \textit{React} a jeho komponenty, které lze znovu využít
         a jednoduše rozšířit. Při vytváření datového modelu pro backend jsme mysleli na modularitu a snažili se o co
         největší nezávislost mezi jednotlivými moduly.
    \item Rozšiřitelnost aplikace.

          S modularitou se pojí i jednoduchá rozšiřitelnost aplikace. Díky modulárnímu návrhu i implementaci je jednoduché
          rozšířit naší aplikaci.
    \item Správa uživatelů, možnost jejich registrování a přihlášení.

          Společně se správou uživatelů jsme navrhli a implementovali i role, které nám mají pomoci zjednodušit správu oprávnění
          pro dokumenty i repozitáře. Role nám pomohou v dalších rozvojích aplikace s rozdělením uživatelů do určitých skupin.
    \item Oprávnění pro dokumenty i moduly.

          Oprávnění jsme již i v tomto závěrečném zhodnocení několikrát zmínili. V rámci implementace bylo potřeba dát si pozor
          na správné filtrování, tuto část aplikace jsme museli pečlivě otestovat.
    \item Rozdělení aplikace na backend a frontend.

          Aplikace jsme již v návrhu rozdělili na frontend a backend a proto máme možnost vytvořit jednoduše například mobilní
          aplikaci, neboť hlavní komunikace probíhá s \gls{rest} \gls{api} na backendu a obě části jsou na sobě více či méně nezávislé.
\end{enumerate}