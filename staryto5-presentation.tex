% arara: xelatex: {options: [-output-directory=build, --shell-escape]}

\documentclass[czech,aspectratio=169]{beamer}

\usepackage{polyglossia}
\setmainlanguage{czech}
\usepackage{ulem}
\usepackage{relsize}
\usepackage{xspace}
\newcommand{\Rplus}{\protect\hspace{-.1em}\protect\raisebox{.35ex}{\smaller{\smaller\textbf{+}}}}
\newcommand{\Cpp}{\mbox{C\Rplus\Rplus}\xspace}

% nastavení vzhledu
% další možnosti vzhledu viz https://hartwork.org/beamer-theme-matrix/
\usetheme{Boadilla}
\usecolortheme{dove}

% vzhled slajdů vnitřní téma (např. vzhled odrážek)
\useinnertheme{circles} %možnosti: default circles rectangles rounded inmargin
% vzhled slajdů vnější téma
\useoutertheme{default} %možnosti: default, miniframes, smoothbars, sidebar, split, shadow, tree, smoothtree, infolines

% zavedeme čvutí modou barvu
\definecolor{CVUT}{HTML}{0065BD}
% čvutí modou použijeme jako hlavní barvu prezentace
\setbeamercolor{structure}{bg=white,fg=CVUT}

% jako font prezentace nadefinujeme oficiální ČVUT písmo Technika
% https://www.cvut.cz/logo-a-graficky-manual  -- inforek, příhlášení přes celoškolské heslo
%\usepackage{fontspec}
%\setsansfont{Technika-Kniha}

% vypneme navigační panel beamer (pro zapnutí zakomentujeme)
\beamertemplatenavigationsymbolsempty{}

% vygenerujeme slajdy s poznámkami
%\setbeameroption{show notes}

% vygeneruje slajdy s poznámky vhodné pro promítání na dvou monitorech
%\usepackage{pgfpages}
%\setbeameroption{show notes on second screen}

% další balíčky
\usepackage[outputdir=build]{minted}
\usepackage{hyperref}
\usepackage{tikz}

% Údaje o prezentaci
\title[Systém pro tvorbu a správu evolvabilních dokumentů]{Systém pro tvorbu a správu evolvabilních dokumentů}
\subtitle{Bakalářská práce}
\institute[FIT ČVUT v Praze]{Fakulta informačních technologií \\ České vysoké učení technické v Praze}
\author[T. Starý]{Tomáš Starý \\ Vedoucí práce: Ing. Marek Suchánek}
\date{3. 5. 2019}
\titlegraphic{\includegraphics[width=.1\textwidth]{logo-cvut}}


\begin{document}

\begin{frame}
    \titlepage{} %generuje se automaticky z section, subsection, subsubsection
    \note{Nezapomenout pozdravit}
\end{frame}

\begin{frame}
    \tableofcontents
\end{frame}

\section{Motivace}
    \begin{frame}
        \tableofcontents[currentsection]
    \end{frame}
    \begin{frame}{Modulární dokumenty}
        \begin{itemize}
            \item rozdělení textu na moduly \pause{}
            \item sdílení modulů \pause{}
            \item snažší správa
        \end{itemize}
    \end{frame}

\section{Nástroje pro tvorbu dokumentů}
    \begin{frame}
        \tableofcontents[currentsection]
    \end{frame}
    \begin{frame}{Markdown}
        \begin{itemize}
            \item značkovací jazyk \pause{}
            \item GitHub, GitLab i Bitbucket \pause{}
            \item neumožňuje modularitu
        \end{itemize}
    \end{frame}
    \begin{frame}{reStructuredText}
        \begin{itemize}
            \item dokumentační nástroj \pause{}
            \item Python \pause{}
            \item umožnuje modularitu \pause{}
            \item .. include::
        \end{itemize}
    \end{frame}
    \begin{frame}{AsciiDoc}
        \begin{itemize}
            \item původně psaný pro Python \pause{}
            \item nyní v jazyce Ruby \pause{}
            \item jako reStructuredText podporuje modularitu \pause{}
            \item include::
        \end{itemize}
    \end{frame}

\section{Aplikace}
    \begin{frame}
        \tableofcontents[currentsection]
    \end{frame}
    \begin{frame}{Nárvh}
        \begin{itemize}
            \item uživatelská sekce \pause{}
            \item repozitáře \pause{}
            \item dokuemnty
        \end{itemize}
    \end{frame}

\end{document}