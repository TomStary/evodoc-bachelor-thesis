Cílem této práce bylo vytvořit aplikaci pro správu a úpravu modulárních dokumentů. Před vytvořením aplikace byla provedena
analýza tvorby dokumentů, ve které jsme si probrali
rozdíly mezi monolitickým a modulárním dokumentem. Na to navázala analýza nástrojů na vytváření dokumentů. Zaměřili jsme se na značkovací jazyky,
které se dnes používájí při vytváření dokumentace
jako například docutil pro jazyk Python či AsciiDoc pro jazyk Ruby.
Po této analýze přijde na řadu návrh aplikace, který zde již lehce nastíníme.

Aplikace bude řešit oprávnění uživatelů, přístup do
jednotlivých repozitářů, což jsou složky s jednotlivými moduly. U jednotlivých modulů musíme myslet na jejich verzování, tedy možnost
vracet se zpět v jednotlivých úpravách. Díky tomuto verzování bude možné generování dokumentů i retrospektivně. Dokumenty, které jsou
složeny z jednotlivých modulů, je možné také verzovat a to pomocí revizí, které si budou pamatovat jaké verze modulů byly použity.

Samotná implementace bude rozdělena na 2 části, backend a frontend. První částí je backend, který nám umožnuje přístup k datům pomocí webového aplikačního
prostředí. Backend má také zajišťovat ověřování uživatelů a\linebreak oprávnění a možnost jejich správy. Frontend nám bude zpřístupňovat \mbox{backend}
ve webovém prohlízeči, jedná se ale o samostatnou aplikaci, která pouze komunikuje a získává data z backendu. Dokončenou implementaci podrobíme
testování, zdali vše funguje v souladu s naším návrhem. Závěrem zhodnotíme výslednou aplikaci, jestli se nám podařilo naplnit všechny cíle, které jsme
si vytičili v návrhu.