Cílem této práce je vytvořit aplikaci pro správu a úpravu modulárních dokumentů. Aplikace bude řešit oprávnění uživatelů, přístup do
jednotlivých repozitářů, což jsou složky s jednotlivými moduly. U jednotlivých modulů musíme myslet na jejich verzování, tedy možnost
vracet se zpět v jednotlivých úpravách. Díky tomuto verzování umožníme generování dokumentů i retrospektivně. Dokumenty, které jsou
složeny z jednotlivých modulů je možné také verzovat a to pomocí revizí, které si budou pamatovat jaké verze modulů byly použity.

Samotná implementace bude rozdělena na 2 části. První částí je backend, který nám umožnuje přístup k datům pomocí webového aplikačního
prostředí. Backend má také zajišťovat ověřování uživatelů a oprávnění a možnost jejich správy. Frontend nám bude zpřístupňovat backend
ve webovém prohlízeči, jedná se ale o samostatnou aplikaci, která pouze komunikuje a získává data z backendu.