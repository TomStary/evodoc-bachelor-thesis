Nástrojů na psaní dokumentů je v dnešní době nepřeberné množství, od placených až po open source řešení.
\todo[inline]{Napsat poznámku k opensource}
\todo{dopsat uvod do kapitoly}

\section{reStructuredText}
reStructuredText je formát psaní dokumentů v prostém textu, který je jednoduchý na čtení, kde je hned zřejmé jak bude vygenerovaný text vypadat.
Tento formát je jednoduchý na použití hlavně pro psaní programátorské dokumentace, jednoduchých webů a samostatných dokumentů.
Hlavním cílem reStructuredText je definovat a uplatnit jednoduchý značkovací jazyk pro použítí v Python docstring a dalších dokumentačních nástrojích,
který je jednoduše čitelnýa jednoduše použitelný. \cite{reStruDoc}

\todo{Doplnit ještě o další pohled z https://www.ibm.com/developerworks/library/x-matters24/}

\section{AsciiDoc}