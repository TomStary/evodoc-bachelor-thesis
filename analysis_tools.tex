Nástrojů jak dnes psát dokumenty máme celou řadu, některé jsou volně dostupné a jiné jsou spoplatněny. Některé nástroje jsme si již představili,
ale nyní se zaměříme na 2 možné nástroje, které můžeme zvolit pro řešení této práce.
\todo[inline]{Napsat poznámku k opensource}

\section{reStructuredText}

reStructuredText je formát psaní dokumentů v prostém textu, který je jednoduchý na čtení, kde je hned zřejmé jak bude vygenerovaný text vypadat.
Tento formát je jednoduchý na použití hlavně pro psaní programátorské dokumentace, jednoduchých webů a samostatných dokumentů.
Hlavním cílem reStructuredText je definovat a uplatnit jednoduchý značkovací jazyk pro použítí v Python docstring a dalších dokumentačních nástrojích,
který je jednoduše čitelnýa jednoduše použitelný. \cite{reStruDoc}

\todo{Doplnit ještě o další pohled z https://www.ibm.com/developerworks/library/x-matters24/}

\section{AsciiDoc}