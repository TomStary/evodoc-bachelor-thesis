Nástrojů jak dnes psát dokumenty máme celou řadu, některé jsou volně dostupné a jiné jsou spoplatněny\todo{ouch}. Některé nástroje jsme si již představili,
ale nyní se zaměříme na 2 možné nástroje, které můžeme zvolit pro řešení této práce.\todo{proč zrovna dva (je to sice v zadání, ale nějak podložit že jsou široce používané a že je to díky jejim vlastnostem)}
\todo[inline]{Napsat poznámku k opensource}

\section{reStructuredText}

reStructuredText\todo{Buď velké R nebo napsat Formát reStructuredText (RST) pro psaní...} je formát psaní dokumentů v prostém textu, který je jednoduchý na čtení, kde je hned zřejmé jak bude vygenerovaný text vypadat.
Tento formát je jednoduchý na použití hlavně pro psaní programátorské dokumentace, jednoduchých webů a samostatných dokumentů.
Hlavním cílem reStructuredText je definovat a uplatnit jednoduchý značkovací jazyk pro použítí v Python docstring\todo{radši vysvětlit co to je docstring - dokumentačních řetězcích, tzv. docstring,} a dalších dokumentačních nástrojích,
který je jednoduše čitelný\todo{mezera}a jednoduše použitelný. \cite{reStruDoc}

\todo{Doplnit ještě o další pohled z https://www.ibm.com/developerworks/library/x-matters24/}

\section{AsciiDoc}