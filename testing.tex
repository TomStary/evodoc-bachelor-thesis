\section{Testování}

V rámci vývoje je také potřeba testovat a zároveň i dokumentovat jednotlivé části. V našem případě při vývoji použijeme
k testování \gls{api} nástroj Postman \cite{postmanSW}, který nám poslouží nejen k dokumentaci našeho backendového \gls{api},
ale také bude fungovat k okamžitému testování našich metod. V Postman je možné definovat jednotlivé dotazy pro náš server s možností
jejich parametrizace a automatického přidání \gls{jwt} tokenu pro ověření uživatele. Mimo možnosti definovat jednotlivé dotazy, lze také vytvořit
pro celé prostředí jednotnou sadu parametrů, které poté mohou být využity v dotazech, ať již ve formě parametrů v \gls{url} nebo také k ukládání
proměnných, které se mohou lišit mezi zařízeními (například adresa testovacího serveru). Na přiloženém médiu je poté kompletní výstup z
tohoto programu. Program je volně ke stažení a je možné sdílet data z jednoho zařízení na další v rámci jednoho účtu, nebo sdílet data
v týmu, tato možnost je již ovšem za poplatek.

Výsledné testování proběhne pomocí akceptačních testů, jedná se o testy,
které jsou provedeny manuálně před samotným nasazením aplikace na produkční prostředí. Testy jsou prováděny podle příkladů užití z návrhu.
Toto testování probíhalo v konfiguraci co nejvíce podobné produkčnímu prostředí, aby nevznikly poté potíže v důsledku jiného prostředí, či
rozdílných konfigurací. Testy se prováděly podle příkladů užití z návrhu. Na základě těchto testů ještě byly provedeny patřičné změny v aplikaci.
Zde je ukázka takového testu pro vytvoření nového uživatelského účtu, příhlášení se a poté také možnosti tento účet upravit.
Nejdříve jsme si vytvořili seznam částí programu, které chceme podrobit testování.
\begin{enumerate}
    \item Uživatel musí mít možnost se zaregistrovat s platnými údaji, pokud údaje jsou neúplné, či neplatné je třeba o tom uživatele informovat.
    \item Registrovaný uživatel je poté schopen se přihlásit do svého nově vytvořeného účtu.
    \item Přihlášený uživatel má možnost měni své údaje.
\end{enumerate}
Na základě tohoto seznamu proběhly kontroly všech zmíněných částí. Problém nastal pouze u změny údajů, kdy se uživateli nezobrazovalo chybové
hlášení při změně údajů na údaje, které ovšem v systému má již jiný uživatel. Podobně byl poté takto zkontrolován celý systém.
A podle jednotlivých průchodý jsme posléze opravili chyby, které se v systému vyskytovali. Po těchto testech je naše aplikace
připravena k nasazení na produkční prostředí.

\section{Nasazení}

Nasazení aplikací se od sebe trochu liší, proto tuto část rozdělíme na 2, ve kterých popíšeme nasazení jednotlivých částí nejenom na produkční
prostředí, ale také jejich vývojové prostředí.

\subsection{Backend}

Pro nasazení backendu použijeme knihovnu pro jazyk Python, \textit{Setuptools} \cite{setupTools}, tato knihovna nám umožní
jednoduše nainstlovat všechny potřebné balíčky, které budeme potřebovat pro běh naší aplikace. Pro nasazení na server
tedy stačí na serveru spustit příkaz \texttt{python setup.py install}, jedinou podmínkou je přitomnost \textit{Setuptools}
na serveru. Před instalací je také dobré nastavit nové připojení na databázi a bezpečnostní klíč. Toto se provede vytvořením
souboru \texttt{config.py} ve složce \texttt{instance}, kterou je potřeba vytvořit v hlavním adresáři aplikace. \textit{Flask} sice nabízí možnost vlastního serveru, jedná se ovšem
pouze o~funkci, která má sloužit vývojářům při vývoji, pro ostré nasazení je toto řešení nestabilní. Proto je potřeba použití WSGI (Web Server Gateway Interface), jedná se o~interface
mezi webovým serverem a webovou aplikací psanou v~jazyce Python. Některé webové servery mají tuto vlastnost již přímo integrovanou (Apache), které WSGI ovšem bude zvoleno
je čistě na správci serveru, na stránkách dokumentace k \textit{Flask} naleznete ukázky pro jednotlivá řešení \cite{flaskDeploy}.


\subsection{Frontend}

Pro frontend využijeme \textit{webpack}, jedná se o modul, který slouží k vytváření balíčků z našeho zdrojového kódu. Tento modul
využijeme již v průběhu vývoje naší aplikace, kdy nám bude kompilovat naše soubory pro lokální použití a také kontroluje importy a exporty
souborů, aby nenastal problém s chybějící referencí. \textit{Webpack} si také umí poradit se soubory ve formátu \gls{scss}, které přeloží
do formátu \gls{css}. Ovšem \textit{webpack} neslouží pouze k obsluze vývojového serveru, ale dá se použít i jako kompilátor pro výsledné
produkční prostředí. Pro tuto vlastnost stačí použít příkaz \texttt{webpack -p --config webpack.config.js}, \textit{webpack} poté vytvoří
ve složce \texttt{dist} výsledný \texttt{main.js} a další soubory. Tyto soubory lze poté nahrát na produkční server, kde aplikace bude
umístěna.
