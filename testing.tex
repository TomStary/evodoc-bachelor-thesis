\section{Testování}

Každou aplikaci je třeba otestovat, ani ta naše není výjímkou. Aplikaci otestujeme pomocí akceptačních testů, jedná se o testy,
které jsou provedeny manuálně před samotným nasazením aplikace na produkční prostředí. Testy jsou prováděny podle příkladů užití z návrhu.
V rámci testování sepíšeme jednodtlivé scénáře, které uživatel má otestovat, na základně výsledků těchto testů poté provedeme potřebné
úpravy aplikace.

\section{Nasazení}

Nasazení aplikací se od sebe trochu liší, proto tuto část rozdělím na 2, ve kterých popíšeme nasazení jednotlivých částí.

\subsection{Backend}

Pro nasazení backendu použijeme knihovnu pro jazyk python, \textit{Setuptools} \cite{setupTools}, tato knihovna nám umožní
jednoduše nainstlovat všechny potřebné balíčky, které budeme potřebovat pro běh naší aplikace. Pro nasazení na server
tedy stačí na serveru spustit příkaz \texttt{python setup.py install}, jedinou podmínkou je přitomnost \textit{Setuptools}
na serveru. Před instalací je také dobré nastavit nové připojení na databázi a bezpečnostní klíč. Toto se provede vytvořením
souboru \texttt{config.py} ve složce \texttt{instance}, kterou je potřeba vytvořit v hlavním adresáři aplikace. Aplikaci je
poté dobré spustit pomocí WSGI serveru \cite{flaskDeploy}.


\subsection{Frontend}

Pro frontend využijeme \textit{webpack}, jedná se o modul, který slouží k vytváření balíčků z našeho zdrojového kódu. Výstupem
\textit{webpack} je JavaScriptový, pro nasazení na produkci použijeme funkci