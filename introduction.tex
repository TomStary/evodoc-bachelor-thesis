S dokumenty se každý z nás setkává každý den, dokonce i teď držíte jeden v ruce. S tím jak se rozvíjela lidská civilizace,
se také rozvíjelo psaní dokumentů, od rytí znaků do krunýřů až po moderní psaní na elektronických zařízení a možnost jejich šíření po internetu.
Ovšem metoda psaní se nezměnila, dokument pro nás stále představuje jednotný celek, představte si ovšem následující případ, máme soubor návodů na náš super
produkt, který má ovšem několik verzí, které se liší i jejich instrukčním manuálem. V našich manuálech se ale nachází odkaz na legislativu, která, jak už to tak
bývá, se mění. Abychom aktualizovali naše návody, musíme změnit všechny soubory, které reprezentují jeden návod. To může být časově náročné a je zde větší riziko
chyby. Cílem této práce je změnit pohled na psaní dokumentů, jednotlivé dokumenty se skládájí z částí, které se vyskytují v různých dokumentech, ale jsou stejné,
jako například náš příklad s návody. Jednotlivé části se pak dají aktualizovat a tyto změny se pak projeví ve všech dokuemntech, změna tedy proběhne
pouze na jednom místě, což sníží riziko chyby a zároveň ušetříme čas.