S dokumenty se každý z nás setkává každý den, dokonce i teď držíte jeden v~ruce. S tím, jak se rozvíjela lidská civilizace,
se také rozvíjelo psaní dokumentů, od rytí znaků do krunýřů až po moderní psaní na elektronických zařízení a možnost jejich šíření po internetu.
Metoda psaní se ale tolik nemě\-nila, dokument pro nás typicky představuje jednotný celek. Představte si ovšem následující případ, máme soubor návodů na náš top
produkt, který má několik verzí, které se liší i jejich instrukčním manuálem. V~našich manuálech se ale nachází odkaz na legislativu, která, jak už to tak
bývá, se mění. Abychom aktualizovali naše návody, musíme změnit všechny soubory, které reprezentují jeden návod. To může být časově náročné a je zde větší riziko
chyby. Tato práce má přispět ke změně pohledu na psaní dokumentů. Jednotlivé dokumenty se~skládájí z částí, které se vyskytují v~různých dokumentech, ale jsou stejné,
jako v~případě našeho příkladu s návody. Jednotlivé části se pak dají aktualizovat a tyto změny se pak projeví ve všech dokumentech, změna tedy proběhne
pouze na jednom místě, což sníží riziko chyby a zároveň se ušetří čas.

S rozdělováním na části se běžně setkáváme v softwarovém inženýrství. Naše aplikace dělíme do jednotlivých vrstev, komponent a ještě více specificky do tříd.
S tímto rozdělováním se musely vypořádat dokumentační nástoroje, které se používají při vytváření dokumentace kódu. V~této práci se na některé podíváme a poté navrhneme řešení,
založené na těchto nástrojích, upravené tak, aby jej bylo možné použít k sestavování dokumentu.