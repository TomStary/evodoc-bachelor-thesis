Implementaci rozdělíme na 2 části a to na část zabívající se backendem, tedy mozkem naší aplikace, která bude zodpovídat
za ukladání a zpracování dat, bude nám také sloužit jako autentifikační server. A poté na frontend, tedy část, která
je zodpovědné za komunikaci s backendem a také bude zpostředkovávat zobrazení dat z backendu uživateli.

\section{Backend}

Jako základ backendu máme Flask, \uv{což je mikroframework pro python\linebreak založený na Werkzeug, Jinja 2 a dobrých záměrů.} \cite{flaskDoc}
Tento mikroframework nabízí vytváření stránek ve formátu \gls{html}, toto ovšem nebude třeba, neboť celý backend je složen pouze z rest metod,
které slouží k získání či úpravě dat. Konkrétně použijeme HTTP metody GET, POST a DELETE, které provedou určitou akci na základě volané adresy.
Pro další funkcnionality využíváme v mnoha případech moduly, které rozšižují základní Flask.

\subsection{Ověřování uživatele}

V rámci ověřovaní uživatelů musíme zajistit zabezpečenou komunikaci mezi backendem a frontendem. Toho docílíme sdílením tokenu, který bude ověřovat
každý dotaz do té části aplikace, kam mají přístup pouze přihlášení uživatelé. Toho docílíme pomocí modulu, který rožšiřujeme Flask,
\textit{Flask-jwt-extended}. Pro správnou funkci tohoto modulu je třeba mít nastavený tajný klíč pro aplikaci, který slouží k šifrování \gls{jwt},
ve kterém se nachází identifikátor uživatele. Dále je potřeba definovat funkce pro načítání uživatele a jeho rolí, toto bude poté použito pro kontrolu
oprávnění uživatele a získávání dat, která jsou vázána na uživatele, zde je ukázka \ref{lst:flaskJWTCode}.

\begin{listing}
    \begin{minted}{python}
        # File: moddoc/service/auth_service.py

        @app.jwt.user_claims_loader
        def add_claims(user):
            """
            Load claims into JWT
            """
            return {'roles': user['roles']}

        @app.jwt.user_identity_loader
        def load_user(user):
            """
            Function for loading user
            """
            return user
    \end{minted}
    \caption{Ukázka kódu pro \textit{flask-jwt-extended}}
    \label{lst:flaskJWTCode}
\end{listing}

\subsection{Propojení s databází}

Na toto použijeme další rozšíření pro \textit{Flask} a to konkrétně \textit{Flask-SQLAlchemy}, jedná se interpretaci \textit{SQLAlchemy}, což je \gls{orm} nástroj pro python.
Podporuje většinu \gls{sql} implementací a serverů. V našem případě budeme brát \textit{postgresql} jako nejvhodnější jazyk a databázový server se zaručenou podporou pro \textit{sqlite}. Protože
v aplikaci používáme jako identifikátor všech modelů používáme \gls{guid}, základní implementace \textit{SQLAlchemy} nám ovšem tento datový typ nativně nepodporuje, stejně jako
některé \gls{sql} servery. Proto si vytvoříme vlastní datový typ \gls{guid}, ukázku jak na to nalezeme i v oficiální dokumentaci \textit{SQLAlchemy} \cite{sqlalchemyGuid}. Jak lze
vidět v ukázce kódu \ref{lst:guidImplementation}, při vytváření tohoto typu je provedena kontrola, na s jakou databází bude aplikace pracovat a podle toho se rozhodne jaký datový typ
bude v rámci databáze použit. Toto se provádí protože ne všechny databáze si umí poradit s GUID datovým formátem.

\begin{listing}
    \begin{minted}{python}
        from sqlalchemy.types import TypeDecorator, CHAR
        from sqlalchemy.dialects.postgresql import UUID
        import uuid

        class GUID(TypeDecorator):
            """Platform-independent GUID type.

            Uses PostgreSQL's UUID type, otherwise uses
            CHAR(32), storing as stringified hex values.

            """
            impl = CHAR

            def load_dialect_impl(self, dialect):
                if dialect.name == 'postgresql':
                    return dialect.type_descriptor(UUID())
                else:
                    return dialect.type_descriptor(CHAR(32))

            def process_bind_param(self, value, dialect):
                if value is None:
                    return value
                elif dialect.name == 'postgresql':
                    return str(value)
                else:
                    if not isinstance(value, uuid.UUID):
                        return "%.32x" % uuid.UUID(value).int
                    else:
                        # hexstring
                        return "%.32x" % value.int

            def process_result_value(self, value, dialect):
                if value is None:
                    return value
                else:
                    if not isinstance(value, uuid.UUID):
                        value = uuid.UUID(value)
                    return value
    \end{minted}
    \caption{Implementace GUID datového typu}
    \label{lst:guidImplementation}
\end{listing}

\clearpage

\subsection{Kontrola příchozích dat}

Jako v každém programu je potřeba kontrolovat vstupní data, je i v našem případě nutné kontrolovat data, která přijdou z našeho frontendu, nebo například z mobilní aplikace,
která se bude připojovat na náš backend. Pro kontrolu dat použijeme implementaci schémat z dalšího modulu pro python, \textit{marshmallow}. Tento modul nám umožní definovat
nejen formát dat, ale také definuje formát výstupních dat, která budou načtena přímo z výstupu dotazu nebo objektu z \textit{SQLAlchemy}.

\section{Frontend}