Implementaci rozdělíme na 2 části a to na část zabívající se backendem, tedy mozkem naší aplikace, která bude zodpovídat
za ukládání a zpracování dat a bude nám také sloužit jako autentifikační server. A poté na frontend, tedy část, která
je odpovědná za komunikaci s backendem a také bude zpostředkovávat zobrazení dat z backendu uživateli.

\section{Backend}

V analýze nástrojů, jsme v samotném závěru porovnávali a také vybírali, jaký značkovací jazyk použijeme jako formát pro naše dokumenty.
Při výčtu výhod pro jazyk reStructuredText, je uvedena autorova předešlá zkušenost s~jazykem Python, tento jazyk se tedy stane jazykem,
který budeme používat při vývoji backendové části naší aplikace.
Dalším důvodem pro výběr tohoto jazyka je modul \textit{pypandoc},
který nám umožňuje převádění reStructuredText na jiné formáty a díky tomuto modulu budeme moci jednoduše generovat naše dokumenty. Mezi jazyky
Python a Ruby nejsou velké rozdíly, oba jsou skriptovací jazyky. Ruby je většinou vnímán pouze jako jazyk na vytváření webových stránek, na druhé straně
Python je zaměřen na univerzálnost jeho použití. \cite{pythonRubyFight}

Jako základ backendu využijeme \textit{Flask}, \uv{což je mikroframework pro Python založený na Werkzeug, Jinja 2 a dobrých záměrů.} \cite{flaskDoc}
Tento mikroframework nabízí vytváření stránek ve formátu \gls{html}, toto ovšem \mbox{nebude} třeba, \mbox{neboť} celý backend je složen pouze z \gls{rest} metod,
které slouží k~získání či úpravě dat. Konkrétně použijeme \gls{http} metody GET, POST a DELETE, které provedou určitou akci na základě volané adresy.
Pro další funkcnionality využíváme v mnoha případech moduly, které rozšižují základní \textit{Flask}. Důvodem pro výběr tohoto frameworku
je jeho snadná rozšiřitelnost pomocí modulů a také nám umožňuje vytvořit jednoduchou webovou aplikaci bez\linebreak zbytečných částí, které by nebyly použity.

\subsection{Ověřování uživatele}

V rámci ověřovaní uživatelů musíme zajistit zabezpečenou komunikaci mezi backendem a frontendem. Toho docílíme sdílením tokenu, který bude ověřovat
každý dotaz do té části aplikace, kam mají přístup pouze přihlášení uživatelé. Toho docílíme pomocí modulu, který rožšiřujeme \textit{Flask},
\textit{Flask-jwt-extended}. Pro správnou funkci tohoto modulu je třeba mít nastavený tajný klíč pro aplikaci, který slouží k šifrování \gls{jwt},
ve kterém se nachází identifikátor uživatele. Dále je potřeba definovat funkce pro načítání uživatele a jeho rolí, toto poté použijeme pro kontrolu
oprávnění uživatele a získávání dat, která jsou vázána na uživatele viz ukázka \ref{lst:flaskJWTCode}. Kromě možnosti sdílení tokenu musíme
také v naší aplikaci vyřešit možnost bezpečného ukládání hesla v naší databázi. Z~bezpečnostního hlediska je nutné zajistit, nemožnost zpětného
zjištění hesla zadaného uživatelem. Toto není úplně možné díky \textit{brute force} útokům, neboli útokům hrubou silou, při které jsou zkoušeny
všechny možné kombinace pro dané heslo. Tomuto se dá lépe předejít silným heslem. Pro šifrování hesla na databázi použijeme modulu \textit{flask-bcrypt},
který implementuje užití hashovací metody bcrypt ve \textit{Flasku}.

\begin{listing}[H]
    \begin{minted}[linenos,breaklines]{python}
# File: moddoc/service/auth_service.py

@app.jwt.user_claims_loader
def add_claims(user):
    """
    Load claims into JWT
    """
    return {'roles': user['roles']}

@app.jwt.user_identity_loader
def load_user(user):
    """
    Function for loading user
    """
    return user
    \end{minted}
    \caption{Ukázka kódu pro \textit{Flask-jwt-extended}}
    \label{lst:flaskJWTCode}
\end{listing}

\subsection{Propojení s databází}

Na toto použijeme další rozšíření pro \textit{Flask} a to konkrétně \textit{Flask-SQLAlchemy}. Jedná se interpretaci \textit{SQLAlchemy}, což je \gls{orm} nástroj pro Python.
Podporuje většinu \gls{sql} implementací a serverů. V našem případě budeme brát \textit{postgresql} jako nejvhodnější databázový server se zaručenou podporou pro \textit{sqlite}. Protože
v aplikaci používáme jako identifikátor všech modelů, používáme \gls{guid}. Základní implementace \textit{SQLAlchemy} nám tento datový typ nativně nepodporuje, stejně jako
některé \gls{sql} servery. Proto si vytvoříme vlastní datový typ \gls{guid}. Ukázku jak na to, nalezeme i v oficiální dokumentaci \textit{SQLAlchemy} \cite{sqlalchemyGuid}. Jak lze
vidět v ukázce kódu \ref{lst:guidImplementation}, při vytváření tohoto typu je provedena kontrola, s jakou databází bude aplikace pracovat a podle toho se rozhodne, jaký datový typ
bude v rámci databáze použit. Toto se provádí proto, že ne všechny databáze si umí poradit s GUID datovým formátem.

\begin{listing}[H]
    \begin{minted}[linenos,breaklines]{python}
#File: moddoc/utils.py
from sqlalchemy.types import TypeDecorator, CHAR
from sqlalchemy.dialects.postgresql import UUID
import uuid

class GUID(TypeDecorator):
"""Platform-independent GUID type.

Uses PostgreSQL's UUID type, otherwise uses
CHAR(32), storing as stringified hex values.

"""
impl = CHAR

def load_dialect_impl(self, dialect):
    if dialect.name == 'postgresql':
        return dialect.type_descriptor(UUID())
    else:
        return dialect.type_descriptor(CHAR(32))

def process_bind_param(self, value, dialect):
    if value is None:
        return value
    elif dialect.name == 'postgresql':
        return str(value)
    else:
        if not isinstance(value, uuid.UUID):
            return "%.32x" % uuid.UUID(value).int
        else:
            # hexstring
            return "%.32x" % value.int

def process_result_value(self, value, dialect):
    if value is None:
        return value
    else:
        if not isinstance(value, uuid.UUID):
            value = uuid.UUID(value)
        return value
    \end{minted}
    \caption{Implementace GUID datového typu}
    \label{lst:guidImplementation}
\end{listing}

\clearpage

Další úpravou, kterou provedeme na implementaci
\textit{Flask-SQLAlchemy} je zavedení vlastního základního modelu, ze kterého pak budeme vytvářet všechny ostatní modely. Tento základní model bude mít
v sobě definovaný identifikátor v~GUID formátu, který jsme si zavedli v naší aplikaci. Díky tomu bude mít každý objekt unikátní identifikátor, neboli také primární
klíč. Díky tomu bude snadnější a rychlejší vyhledávání dat v databázi a to s využitím indexů, které si databáze automaticky vytváří. Také si definujeme
rozšíření pro modely, které jim bude přidávat určité atributy. Tyto atributy jsou stejné v některých tabulkách a nechceme je při vytváření nových tabulek znovu
vypisovat a proto vytvoříme \textit{model mixin}. Tato třída v sobě bude obsahovat pouze definice sloupců, které jsou stejné v těchto určitých tabulkách. Pro upřesnění,
bude se jednat o zavedení takzvaného \uv{soft delete} systému, který nám bude sloužit k virtuálnímu mazání objektů, objektu se pouze nastaví příznak, že byl smazán, ale nebude
smazán z databáze, tudíž jej bude možné obnovit. Tato úprava si ale také žádá definovaní upravených databázových dotazů a s tím nám pomůže definice vlastní
\texttt{QueryClass}, třídy zodpovědné za vytváření dotazů v \textit{Flask-SQLAlchemy} na naší databázi. V této třídě a třídách, které dědí tuto třídu
si poté vytvoříme nové metody, které budou připraveny na filtrování dat s příznakem \texttt{deleted}, jako by byly smazané. Ná závěr této části tu máme
ukázku tříd \ref{lst:sqlalchemyExtensions}, které jsme si nyní popisovali.

\begin{listing}[H]
    \begin{minted}[linenos,breaklines]{python}
#Source: http://flask-sqlalchemy.pocoo.org/2.3/customizing/#model-class  # noqa 501
class IdModel(Model):
    @declared_attr
    def id(cls):
        for base in cls.__mro__[1:-1]:
            if getattr(base, '__table__', None) is not None:
                type = sa.ForeignKey(base.id)
                break
        else:
            type = GUID()

        return sa.Column(type, primary_key=True)


class SoftDeleteModel(object):
    created = sa.Column(sa.DateTime, nullable=False, default=datetime.utcnow)
    updated = sa.Column(sa.DateTime, nullable=True, default=None)
    deleted = sa.Column(sa.DateTime, nullable=True, default=None)

    def delete(self):
        self.deleted = datetime.utcnow()


class SoftDeleteQuery(BaseQuery):
    def soft_get(self, id, default=None):
        object = self.get(id)
        if object and object.deleted is None:
            return object
        else:
            return default

    def soft_all(self):
        return self.filter_by(deleted=None).all()
    \end{minted}
    \caption{Ukázka rozšiřujících tříd pro \textit{Flask-SQLAlchemy}}
    \label{lst:sqlalchemyExtensions}
\end{listing}

\subsection{Kontrola příchozích dat}

Jako v každém programu je potřeba kontrolovat vstupní data, je také v našem případě nutné kontrolovat data, která přijdou z našeho frontendu, nebo\linebreak například z mobilní aplikace,
která se bude připojovat na náš backend. Pro kontrolu dat použijeme implementaci schémat z dalšího modulu pro Python, \textit{marshmallow}. Tento modul nám umožní definovat
nejen formát dat, ale také definuje formát výstupních dat, která budou načtena přímo z~výstupu dotazu nebo objektu z~\textit{SQLAlchemy}.

\subsection{Generování dokumentů}

Asi nejdůležitější částí je generování samotných dokumentů. Jak již bylo\linebreak v~textu této práce zmíněno, bude využit software Pandoc. \cite{pandocSW}, Pro komunikaci s tímto 
softwarem využijeme modul \textit{pypandoc}, který nám poskytne \gls{api} rozhraní pro komunikaci s Pandoc. Pro vytvoření našeho dokumentu již máme skoro vše potřebné. Nyní
je potřeba pouze zajistit zdrojové soubory, ze kterých by se mohl výsledný dokument skládat. Protože si u~každého dokumentu pamatujeme, jaké repozitáře jsou použity, můžeme
tyto repozitáře projít a pro každý modul v nich obsažený vytvořit soubor, který bude obsahovat data z modulu. Název takového souboru se pak bude určovat podle identifikátoru
z databáze pro tento modul. Pokud bude v dokumentu použit odkaz na tento identifikátor, bude zaručeno, že dokument bude generován vždy s nejnovější verzí modulu. Pro generování
dokumentu se specifickou verzí modulu, je třeba uvést jako odkaz identifikátor této verze. Před každým generování dokumentu se nejdříve vytvoří všechny potřebné soubory z modulů,
které mohou být použity v dokumentu a poté se už provede samotné generování dokumentu, který je poté vrácen skrze \gls{api}.

\clearpage

\subsection{Struktura backendu}

Přehlednost kódu je jedna z důležitých vlastností aplikace, kterou je snadné udržovat a rozšiřovat. Tomu může i velice pomoct přehledná a intuitivní
adresářová struktura \ref{lst:backendDir}.
V rámci backendu máme ještě možnost rozdělit si tuto strukturu podle jednotlivých vrstev, které se v naší aplikaci objevují. Napří\-klad databázové modely mohou být ve vlastní složce,
nebo všechny \gls{api} metody lze přesunout do jedné složky a rozdělit do souborů, podle toho, k~jaké entitě se dané metody vážou.

\begin{listing}[H]
    \dirtree{%
        .1 instance\DTcomment{složka s konfigurací, jedná se pouze o lokální složku}.
        .1 migrations\DTcomment{složka s migracemi databáze}.
        .1 moddoc.
        .2 api\DTcomment{složka obahující všechny \gls{api} metody}.
        .2 dto\DTcomment{objekty pro přesun informací mezi databází a uživatelem}.
        .2 model\DTcomment{entity}.
        .2 seed\DTcomment{obsahuje základní set dat pro databázi}.
        .2 service\DTcomment{pomocné služby}.
        .2 \_\_init\_\_.py\DTcomment{základní soubor, obsahuje vytvoření instance Flasku}.
        .2 utils.py\DTcomment{utility použiváné v aplikaci}.
    }
    \caption{Popis strutkury adresářů pro backend}
    \label{lst:backendDir}
\end{listing}

\section{Frontend}

Jako jazyk, ve kterém budeme vyvíjet frontend, volíme JavaScript, který je podporován všemi webovými prohlížeči a umožňujeme nám vytvářet aplikace
přímo v něm. Další jeho přednosti jsou jednoduchost implementace, efektivita a bezpečnost. S popularitou tohoto jazyka roste počet framewroků a knihoven,
které lze použít pro vytváření webových aplikací. Dnes mezi nejznámější patří \textit{VueJS}, \textit{React} a \textit{Angular}. Pouze \textit{Angular} je ovšem plnohodnotný framework, zbylé
2 jsou pouze knihovnami. \cite{jsComparasion} Pro naší aplikaci zvolíme ovšem \textit{React} a to z několika důvodů:
\begin{enumerate}
    \item jednoduchý a lehce pochopitelný
    \item snadná rozšiřitelnost, díky velkému množství balíčků
    \item velká komunita a s tím spojená dobrá dokumentace
    \item jednoduchá možnost dalšího rozšiřování již napsané aplikace, díky znovuvyužití již napsaných komponent.
\end{enumerate}

\textit{React} je, jak již bylo řečeno výše, knihovna psaná v JavaScriptu pro vytvá\-ření webových aplikací. Tato knihovna nám zajištuje základ pro vytváření webového grafického rozhraní.
\textit{React} rozděluje
jednotlivé prvky z rozhraní do komponent, které se poté dají kombinovat, znovu využívat a také se dá u nich definovat rozšiřující chování. Na psaní v \textit{Reactu} není potřeba
definování
si vlastních šablon. Vše je psáno přímo v JavaScriptovém kódu. \cite{reactJS} Samotný \textit{React} poté ještě rozšíříme o další knihovny, které nám usnadní práci s daty, přidají
další komponenty, abychom je nemuseli znovu vytvářet.

\subsection{Redux}

Nejdůležitějším rozšířením \textit{Reactu} v této práci je knihovna \textit{Redux}. \textit{Redux} je, stejně jako \textit{React}, také knihovna pro JavaScript. Jeho hlavní funkcí je
ukládání si stavů aplikace, tudíž je možné předem nadefinovat všechny stavy, ve kterých se naše aplikace může objevit a není problém se pohybovat na časové ose
díky této vlastnosti. Je dobré podotknout, že tyto stavy jsou perzistentní. Pro propojení s \textit{Reactem} existuje knihovna \textit{React-Redux} přímo od vývojářů \textit{Reduxu}. \cite{redux}
Díky tomuto je v \textit{Reactu} poté možné znovu načítat pouze jednotlivé komponenty a není potřeba načítat celou stránku znovu, což má za následek rychlejší odezvu stránek.
Propojení \textit{React} komponenty s \textit{Redux} je jednoduché. Stačí definovat takzvaný \textit{reducer}, neboli definici změny stavů v~závislosti na akcích. Toto je
následně jednoduše přemapováno na \textit{props}, což jsou vstupní hodnoty pro komponety. Komponenty mají přehled o aktuálním stavu aplikace a při změně se znovu vykreslují.

\subsection{Reactstrap a formuláře}

Pro frontend je táké důležité, aby všechny ovládácí prvky byly stejné a celá aplikace měla jednotný design. \textit{Reactstrap} je knihovna rozšiřující \textit{React}
o komponenty,
které jsou již nastylované a to za použití \textit{Bootstrap} stylů. Tyto komponenty jsou následně použity v celé aplikaci a díky Bootstrapu nemusíme vytvářet vlastní styly,
pokud si vystačíme s tím, co nám nábízí \textit{Bootstrap}. \textit{Reactstrap} nabízí sadu stylů pro formuláře. Jejich validace bez odeslání dat by byla náročná, proto
využijeme balíčku \textit{availity-reactstrap-validation}, který obaluje \textit{Reactstrap} vlastními komponenty. Tyto komponenty nabízejí jednoduchou formou možnost
definovat pravidla, která musí platit pro jednotlivá pole ve formulářích. Zde \ref{lst:loginForm} je menší ukázka formuláře z komponenty pro přihlášení. Prvky \texttt{AvInput}
jsou deklarovány s atributem \texttt{required}, který říká, že dané pole nesmí být prázdné. Pokud by bylo, objeví se zpráva, která je definována pro daný \texttt{AvInput} v
\texttt{AvFeedback}. Definování \texttt{AvGroup} je zde pro to, abychom odlišili jednotlivé části formuláře a mohli pro ně definovat 2 různá chybová hlášení v~závislosti na
tom, kterého vstupu se chyba týká. Pokud by se přesto někdo pokusil odeslat formulář s neuplnými daty, je zde podmínka, která tomu braní a to v metodě \texttt{this.handleSubmit}
na ukázce \ref{lst:conditionForm}.


\begin{listing}[H]
    \begin{minted}[linenos,breaklines]{js}
//File: src/Auth/LoginPage.jsx LoginPage:render
<AvForm name='login' onSubmit={this.handleSubmit}>
<AvGroup>
    <AvInput type="text" name="username" id="loginUsername" placeholder={t("Username or email")} required />
    <AvFeedback>{t("Username or email is required.")}</AvFeedback>
</AvGroup>
<AvGroup>
    <AvInput type="password" name="password" id="loginPassword" placeholder={t("Password")} required />
    <AvFeedback>{t("Password is required.")}</AvFeedback>
</AvGroup>
<Button>{t("Login")}</Button>
</AvForm>
    \end{minted}
    \caption{Přihlašovací formulář}
    \label{lst:loginForm}
\end{listing}

\begin{listing}[H]
    \begin{minted}[linenos,breaklines]{js}
//File: src/Auth/LoginPage.jsx LoginPage
handleSubmit(e, err, val) {
    if (err.length == 0) {
        this.props.login(val);
    }
}
    \end{minted}
    \caption{Podmínka odeslání formuláře}
    \label{lst:conditionForm}
\end{listing}

\subsection{Další moduly}

V rámci aplikace je dobré od jejího začátku myslet na možnou lokalizaci textů a k tomu nám pomůže knihovna \textit{react-i18next}, díky které bude jednoduché překládat
jednotlivé části aplikace. Pro snažší komunikaci s backendem využijeme balíčku \textit{refresh-fetch}. Tento balíček nám poslouží pro jednoduchout implementaci obnovování přístupového
\gls{jwt} tokenu za pomoci obnovovacího tokenu.

\clearpage

\subsection{Struktura aplikace}

Pro snažší orientaci při vytváření aplikace, investujeme chvilku času do definování a vytvoření adresářové struktury \ref{dir:frontStruct}. Ta nám
pomůže jednoduše určit, kde se nacházejí jednotlivé části aplikace. Je dobré rozdělit části týkající se pouze \textit{Reduxu} do vlastní složky, či roztřídit
jednotlivé komponenty podle toho, které části aplikace se týkají.

\begin{listing}[H]
    \dirtree{%
        .1 dist\DTcomment{složka obsahující produkční verzi aplikace}.
        .1 public\DTcomment{složka obsahující HTML s výsledným JavaScriptem}.
        .1 src.
        .2 \_actions\DTcomment{zde se nacházejí všechny akce volané z komponent}.
        .2 \_components\DTcomment{pomocné komponenty}.
        .2 \_constants\DTcomment{konstanty využivané v Reduxu}.
        .2 \_helpers\DTcomment{pomocné funkce a třídy}.
        .2 \_reducers\DTcomment{Redux reducers}.
        .2 \_store\DTcomment{Redux store}.
        .2 App\DTcomment{základní komponenta}.
    }
    \caption{Adresářová struktura frontendové aplikace}
    \label{dir:frontStruct}
\end{listing}