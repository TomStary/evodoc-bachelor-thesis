Implementaci rozdělíme na 2 části a to na část zabívající se backendem, tedy mozkem naší aplikace, která bude zodpovídat
za ukladání a zpracování dat, bude nám také sloužit jako autentifikační server. A poté na frontend, tedy část, která
je zodpovědné za komunikaci s backendem a také bude zpostředkovávat zobrazení dat z backendu uživateli.

\section{Backend}

Jako základ backendu máme Flask, \uv{což je mikroframework pro python založený na Werkzeug, Jinja 2 a dobrých záměrů.} \cite{flaskDoc}
Tento mikroframework nabízí vytváření stránek ve formátu \gls{html}, toto ovšem nebude třeba, neboť celý backend je složen pouze z rest metod,
které slouží k získání či úpravě dat. Konkrétně použijeme HTTP metody GET, POST a DELETE, které provedou určitou akci na základě volané adresy.
Pro další funkcnionality využíváme v mnoha případech moduly, které rozšižují základní Flask.

\subsection{Ověřování uživatele}

V rámci ověřovaní uživatelů musíme zajistit zabezpečenou komunikaci mezi backendem a frontendem. Toho docílíme sdílením tokenu, který bude ověřovat
každý dotaz do té části aplikace, kam mají přístup pouze přihlášení uživatelé. Toho docílíme pomocí modulu, který rožšiřujeme Flask,
\textit{flask-jwt-extended}. Pro správnou funkci tohoto modulu je třeba mít nastavený tajný klíč pro aplikaci, který slouží k šifrování \gls{jwt},
ve kterém se nachází identifikátor uživatele. Dále je potřeba definovat funkce pro načítání uživatele a jeho rolí, toto bude poté použito pro kontrolu
oprávnění uživatele a získávání dat, která jsou vázána na uživatele, zde je ukázka \ref{lst:flaskJWTCode}.

\begin{listing}
    \begin{minted}{python}
        # File: moddoc/service/auth_service.py

        @app.jwt.user_claims_loader
        def add_claims(user):
            """
            Load claims into JWT
            """
            return {'roles': user['roles']}

        @app.jwt.user_identity_loader
        def load_user(user):
            """
            Function for loading user
            """
            return user
    \end{minted}
    \caption{Ukázka kódu pro \textit{flask-jwt-extended}}
    \label{lst:flaskJWTCode}
\end{listing}

\subsection{Propojení s databází}

Na toto použijeme další rozšíření pro Flask a to konkrétně Flask-Sqlalchemy, jedná se interpretaci Sqlalchemy, což je \gls{orm} nástroj pro python. Podporuje většinu SQL
implementací a serverů. V našem případě budeme brát \textit{postgresql} jako nejvhodnější jazyk a databázový server se zaručenou podporou pro \textit{sqlite}. Protože
v aplikaci používáme jako identifikátor všech modelů používáme \gls{guid}

\section{Frontend}