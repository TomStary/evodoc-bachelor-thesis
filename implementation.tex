Implementaci rozdělíme na 2 části a to na část zabívající se backendem, tedy mozkem naší aplikace, která bude zodpovídat
za ukladání a zpracování dat, bude nám také sloužit jako autentifikační server. A poté na frontend, tedy část, která
je zodpovědné za komunikaci s backendem a také bude zpostředkovávat zobrazení dat z backendu uživateli.

\section{Backend}

V analýze nástrojů, jsme v samotném závěru porovnávali a také vybírali, jaký značkovací jazyk použijeme jako formát pro naše dokumenty.
Při výčtu výhod pro jazyk reStructuredText, je uvedena autorova předešlá zkušenost s~jazykem Python, tento jazyk se tedy stane jazykem,
který budeme používat při vývoji backendové části naší aplikace.
Dalším důvodem pro výběr tohoto jazyka je modul \textit{pypandoc},
který nám umožňuje převádění reStructuredText na jiné formáty a díky tomuto modulu budeme moci jednoduše generovat naše dokumenty. Mezi jazyky
Python a Ruby nejsou velké rozdíly, oba jsou skriptovací jazyky. Ruby je většinou vnímán pouze jako jazyk na vytváření webových stránek, na druhé straně
Python se snaží zaměřit na univerzálnost jeho použití. \cite{pythonRubyFight}

Jako základ backendu využijeme \textit{Flask}, \uv{což je mikroframework pro Python založený na Werkzeug, Jinja 2 a dobrých záměrů.} \cite{flaskDoc}
Tento mikroframework nabízí vytváření stránek ve formátu \gls{html}, toto ovšem \mbox{nebude} třeba, \mbox{neboť} celý backend je složen pouze z \gls{rest} metod,
které slouží k~získání či úpravě dat. Konkrétně použijeme \gls{http} metody GET, POST a DELETE, které provedou určitou akci na základě volané adresy.
Pro další funkcnionality využíváme v mnoha případech moduly, které rozšižují základní \textit{Flask}. Důvodem pro výběr tohoto frameworku
je jeho snadná rozšiřitelnost pomocí modulů, také nám umožňuje vytvořit jednoduchou webovou aplikaci bez\linebreak zbytečných částí, které by nebyly použity.

\subsection{Ověřování uživatele}

V rámci ověřovaní uživatelů musíme zajistit zabezpečenou komunikaci mezi backendem a frontendem. Toho docílíme sdílením tokenu, který bude ověřovat
každý dotaz do té části aplikace, kam mají přístup pouze přihlášení uživatelé. Toho docílíme pomocí modulu, který rožšiřujeme Flask,
\textit{Flask-jwt-extended}. Pro správnou funkci tohoto modulu je třeba mít nastavený tajný klíč pro aplikaci, který slouží k šifrování \gls{jwt},
ve kterém se nachází identifikátor uživatele. Dále je potřeba definovat funkce pro načítání uživatele a jeho rolí, toto poté použijeme pro kontrolu
oprávnění uživatele a získávání dat, která jsou vázána na uživatele, zde je ukázka \ref{lst:flaskJWTCode}.

\begin{listing}
    \begin{minted}[linenos,breaklines]{python}
# File: moddoc/service/auth_service.py

@app.jwt.user_claims_loader
def add_claims(user):
    """
    Load claims into JWT
    """
    return {'roles': user['roles']}

@app.jwt.user_identity_loader
def load_user(user):
    """
    Function for loading user
    """
    return user
    \end{minted}
    \caption{Ukázka kódu pro \textit{Flask-jwt-extended}}
    \label{lst:flaskJWTCode}
\end{listing}

\subsection{Propojení s databází}

Na toto použijeme další rozšíření pro \textit{Flask} a to konkrétně \textit{Flask-SQLAlchemy}, jedná se interpretaci \textit{SQLAlchemy}, což je \gls{orm} nástroj pro Python.
Podporuje většinu \gls{sql} implementací a serverů. V našem případě budeme brát \textit{postgresql} jako nejvhodnější jazyk a databázový server se zaručenou podporou pro \textit{sqlite}. Protože
v aplikaci používáme jako identifikátor všech modelů používáme \gls{guid}, základní implementace \textit{SQLAlchemy} nám ovšem tento datový typ nativně nepodporuje, stejně jako
některé \gls{sql} servery. Proto si vytvoříme vlastní datový typ \gls{guid}, ukázku jak na to nalezeme i v oficiální dokumentaci \textit{SQLAlchemy} \cite{sqlalchemyGuid}. Jak lze
vidět v ukázce kódu \ref{lst:guidImplementation}, při vytváření tohoto typu je provedena kontrola, s jakou databází bude aplikace pracovat a podle toho se rozhodne jaký datový typ
bude v rámci databáze použit. Toto se provádí protože ne všechny databáze si umí poradit s GUID datovým formátem.

\begin{listing}
    \begin{minted}[linenos,breaklines]{python}
from sqlalchemy.types import TypeDecorator, CHAR
from sqlalchemy.dialects.postgresql import UUID
import uuid

class GUID(TypeDecorator):
"""Platform-independent GUID type.

Uses PostgreSQL's UUID type, otherwise uses
CHAR(32), storing as stringified hex values.

"""
impl = CHAR

def load_dialect_impl(self, dialect):
    if dialect.name == 'postgresql':
        return dialect.type_descriptor(UUID())
    else:
        return dialect.type_descriptor(CHAR(32))

def process_bind_param(self, value, dialect):
    if value is None:
        return value
    elif dialect.name == 'postgresql':
        return str(value)
    else:
        if not isinstance(value, uuid.UUID):
            return "%.32x" % uuid.UUID(value).int
        else:
            # hexstring
            return "%.32x" % value.int

def process_result_value(self, value, dialect):
    if value is None:
        return value
    else:
        if not isinstance(value, uuid.UUID):
            value = uuid.UUID(value)
        return value
    \end{minted}
    \caption{Implementace GUID datového typu}
    \label{lst:guidImplementation}
\end{listing}

\subsection{Kontrola příchozích dat}

Jako v každém programu je potřeba kontrolovat vstupní data, je i v našem případě nutné kontrolovat data, která přijdou z našeho frontendu, nebo\linebreak například z mobilní aplikace,
která se bude připojovat na náš backend. Pro kontrolu dat použijeme implementaci schémat z dalšího modulu pro Python, \textit{marshmallow}. Tento modul nám umožní definovat
nejen formát dat, ale také definuje formát výstupních dat, která budou načtena přímo z~výstupu dotazu nebo objektu z~\textit{SQLAlchemy}.

\subsection{Generování dokumentů}

Asi nejdůležitější částí je generování samotných dokumentů. Jak již bylo v textu této práce zmíněno, bude využit software Pandoc \cite{pandocSW}, abychom mohli s tímto softwarem
komunikovat, využijeme modul \textit{pypandoc}, který nám poskytne \gls{api} rozhraní pro komunikaci s Pandoc. Pro vytvoření našeho dokumenty již máme skoro vše potřebné, nyní
je potřeba pouze zajistit zdrojové soubory, ze kterých by se mohl výsledný dokument skládat. Protože si u každého dokumentu pamatujeme, jaké repozitáře jsou použity, můžeme
tyto repozitáře projít a pro každý modul v nich obsažený vytvořit soubor, který bude obsahovat data z modulu. Název takového souboru se pak bude určovat podle identifikátoru
z databáze pro tento modul. Pokud bude v dokumentu použit odkaz na tento identifikátor bude zaručeno, že dokument bude generován vždy s nejnovější verzí modulu. Pro generování
dokumentu se specifickou verzí modulu, je třeba uvést jako odkaz identifikátor této verze. Před každým generování dokumentu se nejdříve vytvoří všechny potřebné soubory z modulů,
které mohou být použity v dokumentu a poté se už provede samotné generování dokumentu, který je poté vrácen skrze \gls{api}.

\clearpage

\subsection{Struktura backendu}

Přehlednost kódu je jedna z důležitých vlastností aplikace, kterou je snadné udržovat a rozšiřovat. Tomu může i velice pomoct přehledná a intuitivní adresářová struktura.
V rámci backendu máme ještě možnost rozdělit si tuto strukturu podle jednotlivých vrstev, které se v naší aplikaci objevují. Například databázové modely mohou být ve vlastní složce,
nebo všechny \gls{api} metody lze přesunout do jedné složky a rozdělit do souborů, podle toho, k jaké entitě se dané metody vážou.

\begin{listing}[H]
    \dirtree{%
        .1 instance\DTcomment{složka s konfigurací, jedná se pouze o lokální složku}.
        .1 migrations\DTcomment{složka s migracemi databáze}.
        .1 moddoc.
        .2 api\DTcomment{složka obahující všechny \gls{api} metody}.
        .2 dto\DTcomment{objekty pro přesun informací mezi databází a uživatelem}.
        .2 model\DTcomment{entity}.
        .2 seed\DTcomment{obsahuje základní set dat pro databázi}.
        .2 service\DTcomment{pomocné služby}.
        .2 \_\_init\_\_.py\DTcomment{základní soubor, obsahuje vytvoření instance Flasku}.
        .2 utils.py\DTcomment{utility použiváné v aplikaci}.
    }
\end{listing}


\section{Frontend}

Jako jazyk, ve kterém budeme vyvíjet frontend, volíme JavaScript, který je podporován všemi webovými prohlížeči a umožňujeme nám vytvářet aplikace
přímo v něm. Další jeho přednosti jsou jednoduchost implementace, efektivita a bezpečnost. S popularitou tohoto jazyka roste počet framewroků a knihoven,
které lze použít pro vytváření webových aplikací. Dnes mezi nejznámější patří \textit{VueJS}, \textit{React} a \textit{Angular}. Pouze \textit{Angular} je ovšem plnohodnotný framework, zbylé
2 jsou pouze knihovnami. \cite{jsComparasion} Pro naší aplikaci zvolíme ovšem React a to z několika důvodů:
\begin{enumerate}
    \item jednoduchý a lehce pochopitelný
    \item snadná rozšiřitelnost, díky velkému množství balíčků
    \item velká komunita a s tím spojená dobrá dokumentace
    \item jednoduchá možnost dalšího rozšiřování již napsané aplikace, díky znovuvyužití již napsaných komponent.
\end{enumerate}

\textit{React} je, jak již bylo řečeno výše, knihovna psaná v JavaScriptu pro vytváření webových aplikací. Tato knihovna nám zajištuje základ pro vytváření webového grafického rozhraní.
\textit{React} rozděluje
jednotlivé prvky z rozhraní do komponent, které se poté dají kombinovat, znovu využívat a dá se u nich definovat rozšiřující chování. Na psaní v \textit{Reactu} není potřeba psaní
si vlastních šablon, vše je psáno přímo v JavaScriptovém kódu. \cite{reactJS} Samotný React poté ještě rozšíříme o další knihovny, které nám usnadní práci s daty, přidají
další komponenty, abychom je nemuseli znovu vytvářet sami a také zajistí správnou komunikaci s backendem.

\subsection{Redux}

Tím asi nejdůležitějším rozšířením Reactu v této práci je knihovna \textit{Redux}. \textit{Redux} je, stejně jako \textit{React}, také knihovna pro JavaScript. Jeho hlavní funkcí je
ukládání si stavů aplikace, tudíž je možné předem nadefinovat všechny stavy, ve kterých se naše aplikace může objevit a není problém se pohybovat na časové
díky této vlastnosti. Je dobré podotknout, že tyto stavy jsou perzistentní. Pro propojení s Reactem existuje knihovna \textit{React-Redux} přímo od vývojářů \textit{Reduxu}. \cite{redux}
Díky tomuto je v \textit{Reactu} poté možné znovu načítat pouze jednotlivé komponenty a není potřeba načítat celou stránku znovu, což má za následek rychlejší odezvu stránek.
Propojení \textit{React} komponenty s \textit{Redux} je jednoduché, stačí definovat takzvaný \textit{reducer}, neboli definici změny stavů v závislosti na akcích. Toto je
poté jednoduše přemapováno na \textit{props}, což jsou vstupní hodnoty pro komponety, komponenty mají tedy přehled o aktuálním stavu aplikace a při změně se znovu vykreslují.

\subsection{Reactstrap a formuláře}

Pro frontend je táké důležité, aby všechny ovládácí prvky byly stejné a celá aplikace měla jednotný design. \textit{Reactstrap} je knihovna rozšiřující \text{React} o komponenty,
které jsou již nastylované a to za použití \textit{Bootstrap} stylů. Tyto komponenty jsou poté použity v celé aplikaci a díky Bootstrapu nemusíme vytvářet vlastní styly,
pokud si vystačíme s tím, co nám nábízí \textit{Bootstrap}. \textit{Reactstrap} také nabízí sadu stylů pro formuláře, jejich validace bez odeslání dat by byla náročná, proto
využijeme balíčku \textit{availity-reactstrap-validation}, který obaluje \textit{Reactstrap} komponenty vlastními. Tyto komponenty poté nabízejí jednoduchou formou možnost
definovat pravidla, která musí platit pro jednotlivé pole ve formulářích, zde je menší ukázka formuláře z komponenty pro přihlášení \ref{lst:loginForm}. Prvky \texttt{AvInput}
jsou deklarovány s atributem \texttt{required}, který říká, že dané pole nesmí být prázdné, pokud by bylo, objeví se zpráva, která je definována pro daný \texttt{AvInput} v
\texttt{AvFeedback}. Definování \texttt{AvGroup} je zde pro to, abychom odlišili jednotlivé části formuláře a mohli pro ně definovat 2 různá chybová hlášení v závislosti na
tom, kterého vstupu se chyba týká. Pokud by se přesto někdo pokusil odeslat formulář s neuplnými daty, je zde podmínka, která tomu braní a to v metodě \texttt{this.handleSubmit}
na ukázce \ref{lst:conditionForm}.


\begin{listing}[H]
    \begin{minted}[linenos,breaklines]{js}
//File: src/Auth/LoginPage.jsx LoginPage:render
<AvForm name='login' onSubmit={this.handleSubmit}>
<AvGroup>
    <AvInput type="text" name="username" id="loginUsername" placeholder={t("Username or email")} required />
    <AvFeedback>{t("Username or email is required.")}</AvFeedback>
</AvGroup>
<AvGroup>
    <AvInput type="password" name="password" id="loginPassword" placeholder={t("Password")} required />
    <AvFeedback>{t("Password is required.")}</AvFeedback>
</AvGroup>
<Button>{t("Login")}</Button>
</AvForm>
    \end{minted}
    \caption{Přihlašovací formulář}
    \label{lst:loginForm}
\end{listing}

\begin{listing}[H]
    \begin{minted}[linenos,breaklines]{js}
handleSubmit(e, err, val) {
    if (err.length == 0) {
        this.props.login(val);
    }
}
    \end{minted}
    \caption{Podmínka odeslání formuláře}
    \label{lst:conditionForm}
\end{listing}

\subsection{Další moduly}

V rámci aplikace je dobré od jejího začátku myslet na možnou lokalizaci textů, k tomu nám pomůže knihovna \textit{react-i18next}, díky které bude jednoduché překládat
jednotlivé části aplikace. Pro snažší komunikaci s backendem využijeme balíčku \textit{refresh-fetch}, tento balíček nám poslouží pro jednoduchout implementaci obnovování přístupového
\gls{jwt} tokenu za pomoci obnovovacího tokenu.

\subsection{Struktura aplikace}

Aby se při vytváření aplikace dalo snáze orientovat v již napsaném kódu, investujeme chvilku času do definování a vytvoření adresářové struktury \ref{dir:frontStruct}. Ta nám
pomůže jednoduše určit, kde se nacházejí jednotlivé části aplikace, je třeba dobré rozdělit části týkající se pouze Reduxu do vlastní složky, či je dobré roztřídit
jednotlivé komponenty podle toho, které části aplikace se týkají.

\begin{listing}[H]
    \dirtree{%
        .1 dist\DTcomment{složka obsahující produkční verzi aplikace}.
        .1 public\DTcomment{složka obsahující HTML s výsledným JavaScriptem}.
        .1 src.
        .2 \_actions\DTcomment{zde se nacházejí všechny akce volané z komponent}.
        .2 \_components\DTcomment{pomocné komponenty}.
        .2 \_constants\DTcomment{konstanty využivané v Reduxu}.
        .2 \_helpers\DTcomment{pomocné funkce a třídy}.
        .2 \_reducers\DTcomment{Redux reducers}.
        .2 \_store\DTcomment{Redux store}.
        .2 App\DTcomment{základní komponenta}.
    }
    \caption{Adresářová struktura frontendové aplikace}
    \label{dir:frontStruct}
\end{listing}