Práci jsme uvedli historií vývoje písma a vytváření písmeností, prošli jsme si dopad moderních technologií na psaní a šíření dokumentů a představili jsme si
některé programy, které se nyní používají. V následůjicí kapitole jsme si popsali rozdíl mezi monolitickým a modulárním přístupem k psaní dokumentů a také si
načrtli jak probíhá psaní nového dokuemntu.

Po tomto následovalo popsaní značkovacích jazyků, které se čím dál tím více používají i mimo programátorskou komunitu. Zde jsme se také podívali na
\textit{reStructuredText}, který jsme použili posléze v naší implementaci. Tím se dostáváme k realizaci naší aplikace, které byla rozdělena na 2 části
pro lepší možnosti rozšíření, či propojení s případnou mobilní aplikací. V části o backendu jsme se zaměřili na jednotlivé propojení modulů a důvody jejich
použití.

Celou práci zakončíme otestováním naší implementace. Zhodnotíme přínosy naší aplikace a také naše řešení podrobíme porovnání se závěrem z naší analýzy. Jako poslední
část této práce vyjmenujeme možná vylepšení, které by bylo dobré v dalším vývoji implementovat, ať již za účelem zjednodušení testování, či zvýšení výkonu.