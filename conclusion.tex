Práci jsme uvedli historií vývoje písma a vytváření písmeností, prošli jsme dopad moderních technologií na psaní a šíření dokumentů a  představili jsme
některé programy, které se nyní používají. V následůjicí kapitole jsme popsali rozdíl mezi monolitickým a modulárním přístupem k psaní dokumentů a také
načrtli, jak probíhá psaní nového dokumentu.

Popsali jsme značkovací jazyky, které se čím dál tím více používají i mimo programátorskou komunitu. Podívali jsme se na
\textit{reStructuredText}, který jsme posléze použili v naší implementaci. Dostali jsme se tak k realizaci naší aplikace, které byla rozdělena na 2 části
pro lepší možnosti rozšíření, či propojení s případnou mobilní aplikací. V části o backendu jsme se zaměřili na jednotlivé propojení modulů a důvody jejich
použití.

Celou práci jsme zakončili otestováním naší implementace. Zhodnotily přínosy naší aplikace a také jsme naše řešení podrobili porovnání se závěrem z naší analýzy.
Jako poslední
část této práce jsme vyjmenovali možná vylepšení, které by bylo dobré v dalším vývoji implementovat, ať již za účelem zjednodušení testování, či zvýšení výkonu aplikace.