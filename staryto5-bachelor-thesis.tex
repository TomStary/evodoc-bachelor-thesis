% arara: pdflatex: {options: [-output-directory=build, --shell-escape]}
% arara: makeglossaries: {options: [-dbuild]}
% arara: biber: {options: [--output-directory=build]}
% arara: pdflatex: {options: [-output-directory=build, --shell-escape]}

% options:
% thesis=B bachelor's thesis
% thesis=M master's thesis
% czech thesis in Czech language
% slovak thesis in Slovak language
% english thesis in English language
% hidelinks remove colour boxes around hyperlinks

\documentclass[thesis=B,czech]{FITthesis}[2012/06/26]

\usepackage[utf8]{inputenc} % LaTeX source encoded as UTF-8
\usepackage[
	backend=biber
	,style=iso-numeric
	,sortlocale=cs_CZ
	,autolang=other
	,bibencoding=UTF8
]{biblatex}
\addbibresource{sources.bib}

\usepackage{graphicx} %graphics files inclusion
\graphicspath{{./images/}}
% \usepackage{amsmath} %advanced maths
% \usepackage{amssymb} %additional math symbols

\usepackage{dirtree} %directory tree visualisation

% todo package
\usepackage{todonotes}

% code snippets
\usepackage[outputdir=build]{minted}

\usepackage{pdfpages}
\usepackage{listings}

% % list of acronyms
\usepackage[acronym,nonumberlist,toc,numberedsection=autolabel]{glossaries}
\iflanguage{czech}{\renewcommand*{\acronymname}{Seznam pou{\v z}it{\' y}ch zkratek}}{}
\makeglossaries

\newcommand{\tg}{\mathop{\mathrm{tg}}} %cesky tangens
\newcommand{\cotg}{\mathop{\mathrm{cotg}}} %cesky cotangens

% % % % % % % % % % % % % % % % % % % % % % % % % % % % % %
% ODTUD DAL VSE ZMENTE
% % % % % % % % % % % % % % % % % % % % % % % % % % % % % %

\department{Katedra softwarového inženýrství}
\title{Systém pro tvorbu a správu evolvabilních dokumentů}
\authorGN{Tomáš} %(křestní) jméno (jména) autora
\authorFN{Starý} %příjmení autora
\authorWithDegrees{Tomáš Starý} %jméno autora včetně současných akademických titulů
\author{Tomáš Starý} %jméno autora bez akademických titulů
\supervisor{Ing. Marek Suchánek}
\acknowledgements{Doplňte, máte-li komu a za co děkovat. V~opačném případě úplně odstraňte tento příkaz.}
\abstractCS{Práce se zabývá vytvářením dokumentů. Cílem práce je analyzovat tvorbu dokumentů a na základě analýzy vytvořit návrh řešení pro tvorbu
modulárních dokumentů. Na základě tohoto návrhu je následně vytvořena aplikace.
Jedná se o webovou aplikaci, která má oddělené grafické a datové prostředí.
V rámci aplikace je možné vytvářet moduly vně repozitářů, ze kterých je poté možné generovat celé dokumenty, dokumenty i jednotlivé moduly podporují verzování
a je možné získat i dokument starší verze ve správném znění.
Vše je poté otestováno, následuje zhodnocení přínosů systému a porovnání se závěry analýzy, také se podíváme na možnosti jak aplikaci dále rozšiřovat.}
\abstractEN{This thesis will be focused on creating documents. The goal is analyse creation of documents and base on this analysis create
propose application design for creating and managing modular documents. Based on this desing will be then created an application.

This is a web application, which has separate graphic interface and data layer. Repositories and their respective modules are
created within application, from which can documents be generated. Documents and individual modules support versioning and
it is possible to generate documents even its previous version.}
\placeForDeclarationOfAuthenticity{V~Praze}
\declarationOfAuthenticityOption{4} %volba Prohlášení (číslo 1-6)
\keywordsCS{dokumenty, modulární dokumenty, reStructuredText, AsciiDoc, python, ruby, webová aplikace}
\keywordsEN{documents, modular docments, reStructuredText, AsciiDoc, python, ruby}
% \website{http://site.example/thesis} %volitelná URL práce, objeví se v tiráži - úplně odstraňte, nemáte-li URL práce

\renewcommand\listingscaption{Zdrojový kód}
\renewcommand\listoflistingscaption{Seznam zdrojových kódů}

\newglossaryentry{CVUT}
{
	name=ČVUT,
	description={České vysoké učení technické v Praze}
}
\newacronym{FIT}{FIT}{Fakulta informačních technologií}
\newglossaryentry{xml}
{
	name=XML,
	description={Extensible Markup Language}
}
\newglossaryentry{html}
{
	name=HTML,
	description={HyperText Markup Language}
}
\newglossaryentry{pdf}
{
	name=PDF,
	description={Portable Document Format}
}
\newglossaryentry{jwt}
{
	name=JWT,
	description={JSON Web Token}
}
\newglossaryentry{orm}
{
	name=ORM,
	description={Object Relational Mapper, neboli propojení objektů z prostředí programu do tabulek v databázi}
}
\newglossaryentry{guid}
{
	name=GUID,
	description={Globally Unique Identifier, jedná se o unikátní identifikátor, který by měl být naprosto unikátní, jinak se také označují jako UUID (Universaly Unique Identifiers)}
}
\newglossaryentry{gdpr}
{
	name=GDPR,
	description={General Data Protection Regulation v překladu obecné nařízení o ochraně osobních údajů, jedná se legislativu EU, kterou přijala i Česká
	republika. GDPR částečně přepisuje původní znění zákona o ochraně osobních údajů z roku 2000.}
}
\newglossaryentry{sql}
{
	name=SQL,
	description={Structured Query Language, neboli strukturovaný dotazovací jazyk, využívá se v rámci relačních databází k získávání dat.}
}
\newglossaryentry{vcs}
{
	name=VCS,
	description={VCS je zkratkou pro version control systém, což v překladu znamená verzovací systém. Tyto systémy se používájí pro sledování změn
	mezi jednotlivými úpravami souborů.}
}
\newglossaryentry{rest}
{
	name=REST,
	description={Representational State Transfer, jedná se o architekturu, která slouží k vytvoření, čtení, uprávě nebo smazání dat ze serveru pomocí HTTP volání}
}

\begin{document}

\begin{introduction}
	S dokumenty se každý z nás setkává každý den, dokonce i teď držíte jeden v~ruce. S tím, jak se rozvíjela lidská civilizace,
se také rozvíjelo psaní dokumentů, od rytí znaků do krunýřů až po moderní psaní na elektronických zařízení a možnost jejich šíření po internetu.
Metoda psaní se ale tolik nemě\-nila, dokument pro nás typicky představuje jednotný celek. Představte si ovšem následující případ, máme soubor návodů na náš top
produkt, který má několik verzí, které se liší i jejich instrukčním manuálem. V~našich manuálech se ale nachází odkaz na legislativu, která, jak už to tak
bývá, se mění. Abychom aktualizovali naše návody, musíme změnit všechny soubory, které reprezentují jeden návod. To může být časově náročné a je zde větší riziko
chyby. Tato práce má přispět ke změně pohledu na psaní dokumentů. Jednotlivé dokumenty se~skládájí z částí, které se vyskytují v~různých dokumentech, ale jsou stejné,
jako v~případě našeho příkladu s návody. Jednotlivé části se pak dají aktualizovat a tyto změny se pak projeví ve všech dokumentech, změna tedy proběhne
pouze na jednom místě, což sníží riziko chyby a zároveň se ušetří čas.

S rozdělováním na části se běžně setkáváme v softwarovém inženýrství. Naše aplikace dělíme do jednotlivých vrstev, komponent a ještě více specificky do tříd.
S tímto rozdělováním se musely vypořádat dokumentační nástoroje, které se používají při vytváření dokumentace kódu. V~této práci se na některé podíváme a poté navrhneme řešení,
založené na těchto nástrojích, upravené tak, aby jej bylo možné použít k sestavování dokumentu.
\end{introduction}

\chapter{Cíl práce}
Cílem této práce bylo vytvořit aplikaci pro správu a úpravu modulárních dokumentů. Před vytvořením aplikace byla provedena
analýza tvorby dokumentů, ve které jsme si probrali
rozdíly mezi monolitickým a modulárním dokumentem. Na to navázala analýza nástrojů na vytváření dokumentů. Zaměřili jsme se na značkovací jazyky,
které se dnes používájí při vytváření dokumentace
jako například docutil pro jazyk Python či AsciiDoc pro jazyk Ruby.
Po této analýze přijde na řadu návrh aplikace, který zde již lehce nastíníme.

Aplikace bude řešit oprávnění uživatelů, přístup do
jednotlivých repozitářů, což jsou složky s jednotlivými moduly. U jednotlivých modulů musíme myslet na jejich verzování, tedy možnost
vracet se zpět v jednotlivých úpravách. Díky tomuto verzování bude možné generování dokumentů i retrospektivně. Dokumenty, které jsou
složeny z jednotlivých modulů, je možné také verzovat a to pomocí revizí, které si budou pamatovat jaké verze modulů byly použity.

Samotná implementace bude rozdělena na 2 části, backend a frontend. První částí je backend, který nám umožnuje přístup k datům pomocí webového aplikačního
prostředí. Backend má také zajišťovat ověřování uživatelů a\linebreak oprávnění a možnost jejich správy. Frontend nám bude zpřístupňovat \mbox{backend}
ve webovém prohlízeči, jedná se ale o samostatnou aplikaci, která pouze komunikuje a získává data z backendu. Dokončenou implementaci podrobíme
testování, zdali vše funguje v souladu s naším návrhem. Závěrem zhodnotíme výslednou aplikaci, jestli se nám podařilo naplnit všechny cíle, které jsme
si vytičili v návrhu.

\chapter{Historie psaní dokumentů}
\section{Nejdříve trocha historie TODO: možná lepší název kapitoly}

Pokud se chceme zabývat psaním dokumentů, měli bychom si nejdříve udělat menší výlet do historie. V dnešní době bereme vytváření
jakýkoliv dokumentů jako samozřejmost, stačí nám k tomu tužka a kus papíru, nebo můžeme použít počítač, či dokonce i chytrý telefon.

Základem každého dokumentu je písmo, písmo jako takové se prvně objevuje už v 7. tisíciletí před naším letopočtem a to v číně,
kde se našli kosterní pozůstatky a blízko nich i krunýře želv, na kterých se našlo první písmo. \cite{EarliestWriting} Poté se písmo rozšiřuje
i ve staré Mezopotámie a ve starém Egyptě, kde se poté také objevuje jeden z prvních přechůdců papíru papyrus. Ten ovšem po čase upadl
v zapomění a nahradil jej pergamen, který se používal až do 18. století, přestože se v období 13. až 14. století objevuje jeho konkurence, papír.

Po celou tuto dobu se k psaní textů používala výhradně, na území Evropy, lidská síla, tedy všechny dokumenty byly psány ručně, pokud chtěl člověk například
kopii oné knihy, bylo nutné najít písaře, který by danou knihu opsal. Toto byl také jeden z důvodů proč nedocházelo k masovému šíření jakýkoliv textů a byla
to výhrada pro bohatší vrstvy společnosti.

Toto se ovšem změnilo s příchodem knihtisku, o který se hlavně zasloužil Johan Gutenberg, který v polovině 15. století významně zdoknalil knihtisk,
který se pak do konce století rozšířil po celé Evropě. To přineslo zlevnění a zrychlení tvorby nových děl a také se knihy začli šířit mezi větší skupinu
obyvatelstva. Ovšem toto rozšíření stále nebylo tak velké jaké máme dnes a to hlavně díky negramotnosti nižších tříd. V 19. století se s rozšiřující gramotností
i nižších tříd objevuje stále větší hlad po tiskovinách, s průmyslovou revolucí se také rozšiřují možnosti tisku, například roku 1799 si nechal N. L. Robert
patentovat vynález stoje na výrobu papíru v tzv. \uv{nekonečném} pásu. \cite{Papir}

V dnešní době ovšem máme každý možnost si doma vytvořit vlastní dokumenty, noviny či knihy, elektronickou cestou. Po příchodu internetu tak nyní máme možnost
své dokumenty psát a sdílet s ostatními a dokonce na nich i společně pracovat ve stejném okamžitu. Jejich reprodukce do fyzické formy také není problém a to díky
domácím tiskárnám, které jsou cenově dostupné. Při větším počtu kopií je možnost se obrátit i na specializované firmy, které jsou schopny velkoobjemových tisků
v různých kvalitách.

\chapter{Analýza tvorby dokumentů}
Historii psaní dokumentů jsme si popsali v minulé kapitole, nyní je čas popsat princip, kterým dokumenty vznikají. Vznik dokumentů totiž také prochází
vývojem, ale tento vývoj přichází až v poslední době s rozvojem informačních technologí. Tvorba dokumentů by se dala rozdělit na dva různé přístupy.
Jeden má za výsledek jeden soubor, který tvoří daný dokument a druhý pouze složí dokument z určitých částí.

Než si ovšem popíšeme tyto dva přístupy, je nutné se podívat jak vlastně dokument vzniká. Nejdříve je nutné vytvořit návrh neboli se také používá
označení draft. Tento draft je pouze základní obrys toho co by měl výsledný dokument obsahovat. Pokud se na dokumentu podílí vícero autorů, je
tento draft kontrolován každým z nich. Z draftu se potom začne rozvíjet výsledný dokument, který se po dopsání předává ke kontrole korektorům,
kde se kontroluje pravopis a stylistika. Po konečné kontrole díla autorem či autory je dílo předáno distributorovi. Celý tento postup
je znázorněn na grafu \ref{fig:linflow}

\begin{figure}[h]
    \centering
    \includegraphics[width=\textwidth]{linearni_prubeh.png}
    \caption{Swimlines diagram}
    \label{fig:linflow}
\end{figure}

\section{Monolitický přístup}

Monolitickým přístupem je myšleno to, že jednotlivé dokumenty jsou jednotné celky, jeden dokument je jeden celek. Pro představu, pokud si vezmeme ku příkladu
Word, o kterém už zaznělo něco v kapitole o historii psaní dokumentů, vytvořením jednoho souboru s příponou .docx, jsme vytvořili jeden monolitický dokument, pokud jej
budeme chtít použít v jiném dokumentu, budeme muset obsah tohoto souboru zkopírovat do nového souboru. Tento nový soubor, který bude obsahovat i náš původní dokument,
pokud se ovšem něco změní v původním dokumentu, druhý dokument bude mít stále původní verzi. Toto poté přináší problémy, které byly již nastíněny v úvodu této práce.

\section{Modulární přístup}

Hlavní myšlenkou modulárních dokumentů je rozdělit dokument na jednotlivé moduly, tedy části textu, které je možné poté využít i v jiných dokumentech. Jako příklad uvedu
tuto práci, skladá se z jednotlivých kapitol. Tyto kapitoly jsou více či méně na sobě nezávislé a tudíž se dají rozdělit na jednotlivé moduly. Ovšem nemusíme tento
dokument dělit pouze podle kapitol, je možné jej rozdělit i na měnší části. S tímto dělením si ovšem musíme dát pozor, abychom dokument nerozdělili na moc malé části,
které potom samy o sobě nebudou mít moc smysl.

Jak nám toto ale pomůže zlepšit efektivitu psaní a rozšiřování dokumentů? Hlavní je si uvědomit, že pokud se například nový člen týmu, který se má starat o nějakou
dokumentaci, má začlenit do jejího psaní a rozšiřování, je velice těžké se orientovat v jednom velkém monolitickém souboru, pokud ovšem je možnost dokumentaci rozdělit
na části, je i pro nově příchozího jednodušší zorientovat se v této části a případně jí upravit. Pro lidi, kteří dokumentaci píšou již delší dobu, bude mimo jiné přínosné
nutnost držet se jednoho tématu, který může definovat daný modul a tím pádem držet konzistentnost textu. \cite{modularDocuments}


\chapter{Analýza nástrojů pro tvorbu dokumentů}
Nástrojů jak dnes psát dokumenty máme celou řadu, některé jsou volně dostupné a jiné jsou zpoplatněny. Některé nástroje jsme si již představili (Word, LibreOffice), ale
nyní se zaměříme na takzvané značkovací jazyky. Značkovací jazyky nám umožnují psát text, který je poté možné zpracovat počítačem, který změní jeho formátování. Většina
značkovacích jazyků má jasně\linebreak rozlišitelné značky, nebo jinak také tagy, které upravují formátování při jejich strojovém překladu, díky tomu je původní text stále dobře čitelný.
\cite{markup} Výhodou je u nich, kromě jednoduché čitelnosti původního textu, volnost užití, není potřeba pro jejich úpravu žádný speciální software. Pokud je chceme převést do
konečné podoby, stačí nám k tomu volně dostupné nástroje, které jsou ve většině případů dostupné i online a není třeba je tedy instalovat na naše zařízení.

Mezi značkovací jazyky například patří i jazyky používané při\linebreak psaní webových stránek, jako je \gls{html} či \gls{xml}, ovšem tyto jazyky budeme spíše využívat pro zobrazování výstupu
jednodušších značkovacích jazyků. My se budeme hlavně zabývat těmito 3 jazyky: Markdown, reStructuredText a AsciiDoc. Tyto jazyky se používají pro psaní dokumentace v jednotlivých programech
(docutil, AsciiDoc) či slouží k vytváření uživatelských návodů (Markdown).
\clearpage

\section{Markdown}

Markdown je značkovací jazyk, který se typicky převádí do \gls{html}. Jedná se o jednoduše čitelný a zároveň jednoduchý jazyk na psaní strukturovaného textu. Hlavní myšlenkou Markdown je, že
text v něm psaný by měl být publikovatelný i bez jeho zpracování, inspirací tomuto přístupu jsou čistě textové emaily (emaily se dnes ve většině případů píšou v \gls{html}
z důvodu grafického obsahu). \cite{markdown}

Markdown se používá na některých verzovacích službách jako je například GitHub, ovšem v tomto případě se jedná o upravenou implementaci Markdown, která je rozšířena o další
tagy/značky. Podobnou úpravu Markdown má i konkurenční služba GitLab.

Takto vypadá menší ukázka Markdown syntaxe \ref{lst:markdown} a také výsledek, který je poté generován \ref{fig:markdown}.

\begin{listing}[ht]
    \inputminted[linenos,breaklines]{md}{example.md}
    \caption{Příklad Markdown syntaxe}
    \label{lst:markdown}
\end{listing}

\clearpage

\section{reStructuredText}

Formát reStructuredText pro psaní dokumentů v prostém textu, který je snadný na čtení, kde je hned zřejmé, jak bude vygenerovaný text vypadat.
Tento formát je jednoduchý na použití hlavně pro psaní programátorské dokumentace, menších webů a samostatných dokumentů.
Hlavním cílem reStructuredText je definovat a uplatnit jednoduchý značkovací jazyk pro použítí v Python, kde se používá k dokumentaci jednotlivých částí programu,
a dalších dokumentačních nástrojích, který je jednoduše čitelný a jednoduše použitelný. \cite{reStruDoc}

\uv{Docutil je open-source program pro zpracovaní dokumentace v textové podobě do, pro uživatele, přívětivého formátu, jako je například \gls{html}, \LaTeX~či \gls{xml}.} \cite{docutil}
Docutil používá jako vstupní formát již zmíněný reStructuredText. Pro převod do formátu \gls{pdf} lze poté použít utilitu Pandoc \cite{pandocSW}, která umí převěst různé značkovací jazyky
na ostatní formáty jako například náš rst (reStructuredText formát) a to nejenom do \gls{pdf}, ale také do formátů jako je Markdown, \gls{html} a dalších.
Pandoc lze také přímo využít v Pythonu, díky modulu pypandoc \cite{pypandocSW}.

Stejně jako u Markdown, zde máme ukázku syntaxe \ref{lst:rst} i výsledného dokumentu \ref{fig:rstOutput}.

\begin{listing}[ht]
    \inputminted[linenos,breaklines]{rst}{example-rst.rst}
    \inputminted[linenos,breaklines]{rst}{module.rst}
    \caption{Příklad reStructuredText syntaxe}
    \label{lst:rst}
\end{listing}

\clearpage

\section{AsciiDoc}

AsciiDoc je další formát pro psaní dokumentu, jedná se stejně jako reStructuredText, o modul pro jazyk Python. Tento modul je možné použít pro psaní nejenom poznámek,
ale je možné jej exportovat i do formátů jako jsou .epub (formát pro elektronické čtečky knih), či \gls{pdf}. \cite{asciiDoc} Syntaxe jazyku je podobná reStructuredText.

Projekt byl původně psán pro jazyk Python, ovšem posléze byla syntaxe adoptována jako balíček pro jazyk Ruby a také přejmenován na AsciiDoctor \cite{asciiDoctorSW}. AsciiDoctor lze posléze
použít i přímo v kódu, a to konkrétně v jazyce Ruby, zde je menší ukázka nejenom formátu asciiDoc, ale i jeho použití v kódu.

Ukázka syntaxe \ref{lst:asciiDoc} a výsledného dokumentu i pro AsciiDoc je vidět na obrázku \ref{fig:asciiOutput}.

\begin{listing}[ht]
    \inputminted[linenos,breaklines]{text}{example-ascii.adoc}
    \inputminted[linenos,breaklines]{text}{module.adoc}
    \begin{minted}[linenos,breaklines]{ruby}
# example.rb
require 'asciidoctor'

Asciidoctor.convert_file 'example.adoc', to_file: true, safe: :safe
    \end{minted}
    \caption{Příklad AsciiDoc syntaxe a ukázka použití AsciiDoctor}
    \label{lst:asciiDoc}
\end{listing}

\clearpage

\section{Porovnání}

Po představení těchto 3 jazyků je nyní na čase porovnat je, zhodnotit jejich použitelnost pro modulární dokumenty a vybrat si, který použijeme v rámci naší aplikace.
V rámci porovnání musíme dát hlavně důraz na podporu modularity v rámci jazyka, tedy jestli je už v samostatném značkovacím jazyku podpořeno vkládání dalších
souborů s textem, tedy modulů. Dále se také zaměříme na jednoduchost generování různých formátů, některé jazyky jsou podpořeny v Pandocu a pokud jsou, do jakých
formátů je lze převést.

Prvně začněme s jazykem Markdown, jeho představení už máme za sebou a nyní se podívejme na možnost jeho využití při vytváření modulárních dokumentů. Bohužel
Markdown nepodporuje přímé vkládání dalších souborů, tento nedostatek jej činí, oproti ostatním 2 jazykům, nepoužitelným pro modulární dokumenty. Je zde sice možná úprava
kódu, který by text obohatil o moduly před výsledním zpracováním, ale tato úprava by byla náročná na provedí v takovém rozsahu, abychom mohli zaručit její bezchybnost.

Dalším na řadě je reStructuredText, tento jazyk má možnost vkládat v rámci jednoho souboru i další soubory, které obashují další kód, díky této vlastnosti
lze u něho použít modulární přístup k vytváření dokumentů bez žádného předzpracování textu. Tento jazyk je podporován v rámci Pandocu \cite{pandocSW} a je možné
jej převest nejenom do formátu \gls{pdf}, ale i spousty další, jmenovitě například: Markdown, EPUB (formát pro čtečky knih) či již zmíněné \gls{pdf}.

AsciiDoc je poslední jazyk na porovnání, stejně jako reStructuredText, nám i AsciiDoc nabízí možnost připojení dalších souborů, které budou převedeny na výsledný dokument.
Protože je ovšem již podporován pouze v jazyce Ruby, budeme potřebovat na jeho převod do výsledných formátů instalovat jednotlivé moduly pro tento jazyk. Bohužel nelze
vyzužít Pandoc, neboť tuto verzi AsciiDocu již nepodporuje. Použijeme tedy nástroj AsciiDoctor \cite{asciiDoctorSW}, tento nástroj nám výsledný dokument umí převést do
formátu \gls{html}, pro převod do jiných formátů jako například \gls{pdf} je nutné mít kromě AsciiDoctor nainstalováno i jeho rozšíření asciidoctor-pdf \cite{asciidoctorpdfSW}.
Toto trochu znevýhoďnuje AsciiDoc oproti reStructuredText, který má výhodu v možnosti použití jednoho nástroje na převod do vícero formátů.

Na závěr tohoto porovnání je třeba uvést jaký jazyk budeme v naší aplikaci používat. Tímto jazykem bude reStructuredText a to z několika důvodů:
\begin{enumerate}
    \item Prvním důvodem je jeho snadný převod do ostatních formátů díky nástroji Pandoc \cite{pandocSW} a jeho modulu pypandoc \cite{pypandocSW}, který nám umožní používat Pandoc v přímo v aplikaci.
    \item Druhým důvodem je autorova předešlá zkušenost v jazyce Python a také zkušenosti s nástojem Pandoc a modulem pypandoc.
\end{enumerate}

\begin{figure}[h]
    \centering
    \includegraphics[width=\textwidth]{example.pdf}
    \caption{Výstup Markdown}
    \label{fig:markdown}
\end{figure}

\begin{figure}[h]
    \centering
    \includegraphics[width=\textwidth]{example-rst.pdf}
    \caption{Výstup reStructuredText}
    \label{fig:rstOutput}
\end{figure}

\begin{figure}[h]
    \centering
    \includegraphics[width=\textwidth]{example-ascii.pdf}
    \caption{Výstup AsciiDoc}
    \label{fig:asciiOutput}
\end{figure}


\chapter{Návrh}
Nyní je tedy na čase připravit návrh řešení aplikace, která by zajistila správu modulárních dokumentů, jejich generování/publikaci. Nesmíme zapomenout
na důležité funkcionality, které jsme si stanovili v první části této kapitoly. Jedná se hlavně o správu uživatelů, možnost tvořit revize a celkově
verzovat jednotlivé části textu. Dále nesmíme zapomenout na možnost rozšíření aplikace.

\section{Užitelská sekce}

V uživatelské sekci bude hlavní důraz kladen na správu uživatelů, zde také musíme myslet na ochranu jejich osobních udajů a díky GDPR také na
možnost jejich anonymizace.
\todo{Add gdpr stuff}

Uživatele se člověk stává už pouhým otevřením naší aplikace, tento uživatel je nyní v roli hosta, který nemá žádné právo, poté má pouze 2 možnosti, buď se
přihlásí, nebo se registruje jako nový uživatel s právy. Po registraci následuje nutnost potvrdit uživatelský účet jako standardní účet.
Uživatel, stále v roli host, se nyní může přihlásit do aplikace, kde disponuje základními právy, jakými jsou spravování vlastních repozitářů a dokumentů,
či sdílení svých prací s ostatními uživateli se standardními právy. Také zde existuje administrátorská role, která má vyšší než standardní práva,
a to hlavně možnost kontrolovat ostatní uživatele.

O uživatelých si potřebujeme z osobních údajů pamatovat pouze e-mail a uživatelské jméno, které je unikátní pro každého uživatele. Dále si uchováváme
informace o datu založení, poslední změny a smazání uživatele. Dále si pamatujeme role, které uživatel má přiřazené. Role, krom našich základních čtyř
(host, standardní, administrátor, super administrátor), mohou být tvořeny administrátory, kteří je pak mohou přiřadit všem ostatním uživatelům. Podpoření
anonymizace bude založeno na změně uživatelského jména a emailu na sled náhodných znaků, ze kterých nebude patrný původní obsah, tzv. \textit{hash}. Každý z
uživatelů bude mít možnost smazat svůj účet, nebo si jej plně anonymizovat.

Za data, která užvatelé napíší již v rámci dokumentů neneseme odpovědnost, ovšem v případě smluv, či jiných například úředních dokumentů by bylo dobré
zajistit možnost jejich vyplnění až při vystavení dokumentu, aby pak tedy nemuseli být v aplikaci vůbec uloženy.
\todo{Question: Dopsat možnosti parametrizace dat, možná to napsat do analýzy, nebo ještě lépe, konzultovat}

% this might be used, but not now
% \begin{center}
%     \includegraphics[width=8cm]{lifecycle.png}
% \end{center}

\chapter{Realizace}
Implementaci rozdělíme na 2 části a to na část zabívající se backendem, tedy mozkem naší aplikace, která bude zodpovídat
za ukladání a zpracování dat, bude nám také sloužit jako autentifikační server. A poté na frontend, tedy část, která
je zodpovědné za komunikaci s backendem a také bude zpostředkovávat zobrazení dat z backendu uživateli.

\section{Backend}

Jako základ backendu máme Flask, \uv{což je mikroframework pro Python\linebreak založený na Werkzeug, Jinja 2 a dobrých záměrů.} \cite{flaskDoc}
Tento mikroframework nabízí vytváření stránek ve formátu \gls{html}, toto ovšem nebude třeba, \mbox{neboť} celý backend je složen pouze z \gls{rest} metod,
které slouží k získání či úpravě dat. Konkrétně použijeme HTTP metody GET, POST a DELETE, které provedou určitou akci na základě volané adresy.
Pro další funkcnionality\linebreak využíváme v mnoha případech moduly, které rozšižují základní Flask.

\subsection{Ověřování uživatele}

V rámci ověřovaní uživatelů musíme zajistit zabezpečenou komunikaci mezi backendem a frontendem. Toho docílíme sdílením tokenu, který bude ověřovat
každý dotaz do té části aplikace, kam mají přístup pouze přihlášení uživatelé. Toho docílíme pomocí modulu, který rožšiřujeme Flask,
\textit{Flask-jwt-extended}. Pro správnou funkci tohoto modulu je třeba mít nastavený tajný klíč pro aplikaci, který slouží k šifrování \gls{jwt},
ve kterém se nachází identifikátor uživatele. Dále je potřeba definovat funkce pro načítání uživatele a jeho rolí, toto bude poté použito pro kontrolu
oprávnění uživatele a získávání dat, která jsou vázána na uživatele, zde je ukázka \ref{lst:flaskJWTCode}.

\begin{listing}
    \begin{minted}{python}
        # File: moddoc/service/auth_service.py

        @app.jwt.user_claims_loader
        def add_claims(user):
            """
            Load claims into JWT
            """
            return {'roles': user['roles']}

        @app.jwt.user_identity_loader
        def load_user(user):
            """
            Function for loading user
            """
            return user
    \end{minted}
    \caption{Ukázka kódu pro \textit{Flask-jwt-extended}}
    \label{lst:flaskJWTCode}
\end{listing}

\subsection{Propojení s databází}

Na toto použijeme další rozšíření pro \textit{Flask} a to konkrétně \textit{Flask-SQLAlchemy}, jedná se interpretaci \textit{SQLAlchemy}, což je \gls{orm} nástroj pro Python.
Podporuje většinu \gls{sql} implementací a serverů. V našem případě budeme brát \textit{postgresql} jako nejvhodnější jazyk a databázový server se zaručenou podporou pro \textit{sqlite}. Protože
v aplikaci používáme jako identifikátor všech modelů používáme \gls{guid}, základní implementace \textit{SQLAlchemy} nám ovšem tento datový typ nativně nepodporuje, stejně jako
některé \gls{sql} servery. Proto si vytvoříme vlastní datový typ \gls{guid}, ukázku jak na to nalezeme i v oficiální dokumentaci \textit{SQLAlchemy} \cite{sqlalchemyGuid}. Jak lze
vidět v ukázce kódu \ref{lst:guidImplementation}, při vytváření tohoto typu je provedena kontrola, na s jakou databází bude aplikace pracovat a podle toho se rozhodne jaký datový typ
bude v rámci databáze použit. Toto se provádí protože ne všechny databáze si umí poradit s GUID datovým formátem.

\begin{listing}
    \begin{minted}{python}
        from sqlalchemy.types import TypeDecorator, CHAR
        from sqlalchemy.dialects.postgresql import UUID
        import uuid

        class GUID(TypeDecorator):
            """Platform-independent GUID type.

            Uses PostgreSQL's UUID type, otherwise uses
            CHAR(32), storing as stringified hex values.

            """
            impl = CHAR

            def load_dialect_impl(self, dialect):
                if dialect.name == 'postgresql':
                    return dialect.type_descriptor(UUID())
                else:
                    return dialect.type_descriptor(CHAR(32))

            def process_bind_param(self, value, dialect):
                if value is None:
                    return value
                elif dialect.name == 'postgresql':
                    return str(value)
                else:
                    if not isinstance(value, uuid.UUID):
                        return "%.32x" % uuid.UUID(value).int
                    else:
                        # hexstring
                        return "%.32x" % value.int

            def process_result_value(self, value, dialect):
                if value is None:
                    return value
                else:
                    if not isinstance(value, uuid.UUID):
                        value = uuid.UUID(value)
                    return value
    \end{minted}
    \caption{Implementace GUID datového typu}
    \label{lst:guidImplementation}
\end{listing}

\clearpage

\subsection{Kontrola příchozích dat}

Jako v každém programu je potřeba kontrolovat vstupní data, je i v našem případě nutné kontrolovat data, která přijdou z našeho\linebreak frontendu, nebo například z mobilní aplikace,
která se bude připojovat na náš backend. Pro kontrolu dat použijeme implementaci schémat z dalšího modulu pro Python, \textit{marshmallow}. Tento modul nám umožní definovat
nejen formát dat, ale také definuje formát výstupních dat, která budou načtena přímo z~výstupu dotazu nebo objektu z~\textit{SQLAlchemy}.

\section{Frontend}

Základním stavebním kamenem frontendu je knihovna React psaná v JavaScriptu. Tato knihovna nám zajištuje základ pro vytváření webového grafického rozhraní. React rozděluje
jednotlivé prvky z rozhraní do komponent, které se poté dají kombinovat, znovu využívat a dá se u nich definovat\linebreak rozšiřující chování. Na psaní v Reactu není potřeba psaní
si vlastních šablon, vše je psáno přímo v JavaScriptovém kódu. \cite{reactJS} Samotný React ještě rozšíříme o~další důležité knihovny.

\subsection{Redux}

Tím asi nejdůležitějším rozšířením Reactu v této práci je knihovna Redux. Redux je, stejně jako React, také knihovna pro JavaScript. Jeho hlavní funkcí je
ukládání si stavů aplikace, tudíž je možné předem nadefinovat všechny stavy, ve kterých se naše aplikace může objevit a není problém se pohybovat na časové
díky této vlastnosti. Je dobré podotknout, že tyto stavy jsou perzistentní. Pro propojení s Reactem existuje knihovna React-Redux přímo od vývojářů Reduxu. \cite{redux}

\subsection{Reactstrap}

Pro frontend je táké důležité, aby všechny ovládácí prvky byly stejné a celá aplikace měla jednotný design. Reactstrap je knihovna rozšiřující React o komponenty,
které jsou již nastylované a to za použití Bootstrap stylů. Tyto komponenty jsou poté použity v celé aplikaci a díky Bootstrapu nemusíme vytvářet vlastní styly,
pokud si vystačíme s tím, co nám nábízí Bootstrap.

\subsection{Struktura aplikace}

Aby se při vytváření aplikace dalo snáze orientovat v již napsaném kódu, investujeme chvilku času do definování a vytvoření adresářové struktury. Ta nám
pomůže jednoduše určit, kde se nacházejí jednotlivé části aplikace, je třeba dobré rozdělit části týkající se pouze Reduxu do vlastní složky, či je dobré roztřídit
jednotlivé komponenty podle toho, které části aplikace se týkají.


\begin{conclusion}
	Práci jsme uvedli historií vývoje písma a vytváření písmeností, prošli jsme si dopad moderních technologií na psaní a šíření dokumentů a představili jsme si
některé programy, které se nyní používají. V následůjicí kapitole jsme si popsali rozdíl mezi monolitickým a modulárním přístupem k psaní dokumentů a také si
načrtli jak probíhá psaní nového dokuemntu.

Po tomto následovalo popsaní značkovacích jazyků, které se čím dál tím více používají i mimo programátorskou komunitu. Zde jsme se také podívali na
\textit{reStructuredText}, který jsme použili posléze v naší implementaci. Tím se dostáváme k realizaci naší aplikace, které byla rozdělena na 2 části
pro lepší možnosti rozšíření, či propojení s případnou mobilní aplikací. V části o back\-endu jsme se zaměřili na jednotlivé propojení modulů a důvody jejich
použití.

Celou práci zakončíme otestováním naší implementace. Zhodnotíme příno\-sy naší aplikace a také naše řešení podrobíme porovnání se závěrem z naší analýzy. Jako poslední
část této práce vyjmenujeme možná vylepšení, které by bylo dobré v dalším vývoji implementovat, ať již za účelem zjednodušení testování, či zvýšení výkonu.
\end{conclusion}
\listoftodos
\printbibliography
\appendix

\printglossaries

\chapter{Obsah přiloženého CD}

%upravte podle skutecnosti

\begin{figure}
	\dirtree{%
		.1 readme.txt\DTcomment{stručný popis obsahu CD}.
		.1 exe\DTcomment{adresář se spustitelnou formou implementace}.
		.1 src.
		.2 impl\DTcomment{zdrojové kódy implementace}.
		.2 thesis\DTcomment{zdrojová forma práce ve formátu \LaTeX{}}.
		.1 text\DTcomment{text práce}.
		.2 thesis.pdf\DTcomment{text práce ve formátu PDF}.
		.2 thesis.ps\DTcomment{text práce ve formátu PS}.
	}
\end{figure}

\end{document}
